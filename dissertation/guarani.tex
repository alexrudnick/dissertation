\urldef{\leydelenguas}\url{http://www.cultura.gov.py/lang/es-es/2011/05/ley-de-lenguas-n%C2%BA-4251/}

\chapter{Paraguay and the Guarani Language}
Guarani is an indigenous language spoken in Paraguay and the surrounding
region.
Historically, it was the native language of the indigenous Guarani people. The
word for the Guarani language in Guarani is \emph{avañe'e} (``people's
language", where \emph{ñe'e} means ``language").

Guarani is unique among indigenous American languages in that a substantial
number of non-indigenous people speak it.  The majority of Paraguayans are
conversant in Guarani, although they are likely to be bilingual with Spanish.
In practice, many Paraguayans use a combination of Guarani and Spanish called
\emph{Jopar{\'a}}, which is the Guarani word for ``mixture".

Paraguay is officially a bilingual, pluricultural country, as described by its
famous \emph{Ley de Lenguas} \footnote{\leydelenguas} (``Law of Languages").
However, the Guarani language is at a significant social and economic
disadvantage and is typically not used in formal situations, as Spanish is
often considered more prestigious. There are, however, an engaged activist
community, many Guarani-language educators, and a government agency devoted
specifically to policy regarding language.
The Guarani language figures significantly into a sense of Paraguayan national
identity and history.

Guarani has a rich, polysynthetic, agglutinative morphology, in which roots can
derive into different parts of speech, and often several roots can combine into
a single word. Guarani morphology can mark tense, aspect (even on nouns),
number, negation, and other features. However, unlike Spanish, it has no
grammatical gender.  Guarani's rich morphology can make many NLP tasks,
including ones seemingly as simple as spell-checking, rather challenging.

We are in contact with a number of collaborators in Paraguay, including
language activists and educators from the \emph{Ateneo de la Lengua y Cultura
Guaraní} \footnote{\url{http://www.ateneoguarani.edu.py/}} and the
\emph{Fundación Yvy Marãe'{\~y}} \footnote{\url{http://yvymaraey.org/}},
both of which are schools that offer training for Guarani-language translators.

We have also started discussing development plans with several local software
developers -- including some from the local One Laptop Per Child organization
-- interested in building open source software, such as the corpus-building
websites described in the next section.
