\chapter{Related Work}

\section{CL-WSD \emph{in vitro}}
ParaSense, the CL-WSD system developed by Els Lefever
\cite{lefever-hoste-decock:2011:ACL-HLT2011}, takes into account evidence from
several different parallel corpora.
For predicting the translation of a source word into
any particular target language, ParaSense creates
bag-of-words features from the translations of the input sentence into every
other language that it knows about. As of this writing, ParaSense handles
translation from English into French, Spanish, Italian, Dutch and German.
Given corpora that are parallel over many languages, such as Europarl, this is
straightforward at
training time. However, at testing time it requires a complete MT system for
each of the four other languages, which seems computationally prohibitive. In
the past, ParaSense has simply called out to the Google Translate API to
generate the bag-of-words features required for test sentences. This seems
unwieldy, and thus in our work, we learn from several parallel corpora but
require neither a locally running MT system nor access to an online translation
API.


\section{CL-WSD at SemEval}
CL-WSD has received enough attention to warrant shared tasks at recent SemEval
workshops; the most recent running of the task is described by Lefever and
Hoste \cite{task10}.
In this task, participants are asked to translate twenty different polysemous
English nouns into five different European languages, in a variety of contexts.


\section{multilingual evidence for CL-WSD}

%% ParaSense and everything pertaining to ParaSense
Lefever \emph{et al.}, in work on the ParaSense system
\cite{lefever-hoste-decock:2011:ACL-HLT2011}, produced top results for
this task with classifiers trained on local contextual features, with the 
addition of a bag-of-words model of the translation of the complete source
sentence into other (neither the source nor the target) languages. At training
time, the foreign bag-of-words features for a sentence are extracted from
available parallel corpora, but at testing time, they must be
estimated with a third-party MT system, as they are not known a priori.
This work has not yet, to our knowledge, been integrated into an MT system
on its own.

Lefever \textit{et al.} recently described a novel approach to WSD, making use
of evidence from several languages at once to disambiguate English-language
source sentences. This is done by building artificial parallel corpora in
several languages, on demand, with the Google Translate API. They outperform
the previous state-of-the art systems on the SemEval 2010 shared task 13
\cite{lefever-hoste-decock:2011:ACL-HLT2011}.

In our earlier work, we prototyped a system that addresses some of the issues
with ParaSense, requiring more modest software infrastructure for feature
extraction while still allowing CL-WSD systems to make use of several mutually
parallel bitexts that share a source language
\cite{rudnick-liu-gasser:2013:SemEval-2013}.

ParaSense, the CL-WSD system developed by Els Lefever
\cite{lefever-hoste-decock:2011:ACL-HLT2011}, takes into account evidence from
several different parallel corpora.
For predicting the translation of a source word into
any particular target language, ParaSense creates
bag-of-words features from the translations of the input sentence into every
other language that it knows about. As of this writing, ParaSense handles
translation from English into French, Spanish, Italian, Dutch and German.
Given corpora that are parallel over many languages, such as Europarl, this is
straightforward at
training time. However, at testing time it requires a complete MT system for
each of the four other languages, which seems computationally prohibitive. In
the past, ParaSense has simply called out to the Google Translate API to
generate the bag-of-words features required for test sentences. This seems
unwieldy, and thus in our work, we learn from several parallel corpora but
require neither a locally running MT system nor access to an online translation
API.

\section{WSD with sequence models}
To our knowledge, there has not been other work on framing CL-WSD for all words
in an input sentence as a sequence labeling problem. However, in monolingual
WSD, Molina \textit{et al.} \cite{DBLP:conf/iberamia/MolinaPS02} have made
use of HMMs for WSD. 

To our knowledge, there has not been work specifically on sequence labeling
applied to lexical selection for RBMT systems.

Outside of the CL-WSD setting, there has been work on framing all-words WSD as
a sequence labeling problem. Particularly, Molina \textit{et al.}
\cite{DBLP:conf/iberamia/MolinaPS02} have made use of HMMs for all-words
WSD in a monolingual setting.

We have also done some previous work on CL-WSD for translating into indigenous
American languages; an earlier version of Chipa, for Spanish-Guarani, made use
of sequence models to jointly predict all of the translations for a sentence at
once \cite{rudnick-gasser:2013:HyTra}.

XXX:
Look into the recent stuff from BabelNet and Babelfy: what are they doing
there?


\section{CL-WSD for Lexical Selection in RBMT}

Although they did not present a complete MT system, there has also been work
on using WSD techniques for translation into lower-resourced languages, such as
the English-Slovene language pair, as in
\cite{vintar-fivser-vrvsvcaj:2012:ESIRMT-HyTra2012}. 

The Apertium team has a particular practical interest in improving lexical
selection in RBMT; they recently have been developing
a new system, described in \cite{tyers-fst}, that learns finite-state
transducers for lexical selection from the available parallel corpora. It is
intended to be both very fast, for use in practical translation systems, and
to produce lexical selection rules that are understandable and modifiable by
humans.

Francis Tyers, in his dissertation work \cite{tyers-dissertation},
provides an overview of lexical selection systems and describes methods for
learning lexical selection rules based on available parallel corpora.
These rules make reference to the lexical items and parts of speech surrounding
the word to be translated. Once learned, these rules are intended to be
understandable and modifiable by human language experts. For practical use in
the Apertium machine translation system, they are compiled to finite-state
transducers.


Francis Tyers, in his dissertation work \cite{tyers-dissertation},
provides an overview of lexical selection systems and describes methods for
learning lexical selection rules based on available parallel corpora. These
rules make reference to the lexical items and parts of speech surrounding the
word to be translated. Once learned, these rules are intended to be
understandable and modifiable by human language experts. For practical use in
the Apertium machine translation system, they are compiled to finite-state
transducers.

The Apertium team has a particular practical interest in improving lexical
selection in RBMT; they recently have been developing
a new system, described in \cite{tyers-fst}, that learns finite-state
transducers for lexical selection from the available parallel corpora. It is
intended to be both very fast, for use in practical translation systems, and
to produce lexical selection rules that are understandable and modifiable by
humans.


Francis Tyers, in his dissertation work \cite{tyers-dissertation},
provides an overview of lexical selection systems and describes methods for
learning lexical selection rules based on available parallel corpora.
These rules make reference to the lexical items and parts of speech surrounding
the word to be translated. Once learned, these rules are intended to be
understandable and modifiable by human language experts. For practical use in
the Apertium machine translation system, they are compiled to finite-state
transducers.

\section{CL-WSD for PB-SMT}

%% everything about Carpuat and Wu goes here
More recently, Carpuat and Wu have
shown how to use classifiers to improve modern phrase-based SMT systems
\cite{carpuatpsd}.


While most SMT systems do not make use of an explicit WSD module, recently
there has been work on adding WSD classifiers in to statistical MT systems.
Particularly, Carpuat and Wu have shown how to use CL-WSD, or more broadly,
cross-lingual phrase sense disambiguation, to improve modern phrase-based SMT
systems
\cite{carpuatpsd,carpuat-wu:2007:EMNLP-CoNLL2007,carpuat2008evaluation}. In
Carpuat's work, classifiers are used to label multi-word expressions (phrases,
in the phrase-based SMT sense) with target language phrases. She demonstrates
how this is more appropriate in an an SMT setting than simply labeling
individual words with WordNet synsets, as had previously been attempted, and
showed significant improvements on a Chinese-English translation task.

While most SMT systems do not make use of an explicit WSD module, recently
there has been work on adding WSD classifiers in to statistical MT systems.
Particularly, Carpuat and Wu have shown how to use CL-WSD, or more broadly,
cross-lingual phrase sense disambiguation, to improve modern phrase-based SMT
systems
\cite{carpuatpsd,carpuat-wu:2007:EMNLP-CoNLL2007,carpuat2008evaluation}. In
Carpuat's work, classifiers are used to label multi-word expressions (phrases,
in the phrase-based SMT sense) with target language phrases. She demonstrates
how this is more appropriate in an an SMT setting than simply labeling
individual words with WordNet synsets, as had previously been attempted, and
showed significant improvements on a Chinese-English translation task.

There has also been work on using discriminative MaxEnt models to adapt
the ``forward" translation model of an SMT system, using richer
source-language context features \cite{vzabokrtsky-popel-marevcek:2010:WMT}.
While it has much the same effect, this work does not describe itself in terms
of word sense disambiguation.




\section{WSD for lower-resourced languages}
However, there has been work recently on using WSD techniques for translation
into lower-resourced languages, such as the English-Slovene language pair, as
in \cite{vintar-fivser-vrvsvcaj:2012:ESIRMT-HyTra2012}. 

Dinu and Kübler~\cite{Dinu07} have addressed the problem of monolingual WSD for
a lower-resourced language, particularly Romanian. In their work, they describe
an instance-based approach in which a relatively small number of features is
used quite effectively. In other work on lower-resourced languages,
Sarrafzdadeh \textit{et al.} have investigated a version of the Lesk algorithm
for Farsi.

Although they did not present a complete MT system, there has also been work
on using WSD techniques for translation into lower-resourced languages, such as
the English-Slovene language pair, as in
\cite{vintar-fivser-vrvsvcaj:2012:ESIRMT-HyTra2012}. 

%% XXX
One sort of interesting point here is that we are not doing WSD on a
low-resourced language. This is ultimately WSD on Spanish with a sense
inventory that we automatically discover from cross-lingual evidence.


\section{Translation into Morphologically Rich Languages}
Chris Dyer's recent paper at EMNLP
\cite{chahuneau:2013:emnlp}

Talk about prediction for morphology.
\cite{toutanova-suzuki-ruopp:2008:ACLMain}

Also factored models...
\cite{yeniterzi-oflazer:2010:ACL}
