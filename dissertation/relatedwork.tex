\chapter{Related Work}
\label{sec:relatedwork}

\section{CL-WSD \emph{in vitro}}
ParaSense, the CL-WSD system developed by Els Lefever
\cite{lefever-hoste-decock:2011:ACL-HLT2011}, takes into account evidence from
several different parallel corpora.
For predicting the translation of a source word into
any particular target language, ParaSense creates
bag-of-words features from the translations of the input sentence into every
other language that it knows about. As of this writing, ParaSense handles
translation from English into French, Spanish, Italian, Dutch and German.
Given corpora that are parallel over many languages, such as Europarl, this is
straightforward at
training time. However, at testing time it requires a complete MT system for
each of the four other languages, which seems computationally prohibitive. In
the past, ParaSense has simply called out to the Google Translate API to
generate the bag-of-words features required for test sentences. This seems
unwieldy, and thus in our work, we learn from several parallel corpora but
require neither a locally running MT system nor access to an online translation
API.

To our knowledge, there has not been other work on framing CL-WSD for all words
in an input sentence as a sequence labeling problem. However, in monolingual
WSD, Molina \textit{et al.} \cite{DBLP:conf/iberamia/MolinaPS02} have made
use of HMMs for WSD. 

%%\subsection{Translation into Morphologically Rich Languages}
%%Chris Dyer's recent paper at EMNLP
%%\cite{chahuneau:2013:emnlp}
%%Talk about prediction for morphology.
%%\cite{toutanova-suzuki-ruopp:2008:ACLMain}
%%Also factored models...
%%\cite{yeniterzi-oflazer:2010:ACL}

\section{CL-WSD for Lexical Selection}
While most SMT systems do not make use of an explicit WSD module, recently
there has been work on adding WSD classifiers in to statistical MT systems.
Particularly, Carpuat and Wu have shown how to use CL-WSD, or more broadly,
cross-lingual phrase sense disambiguation, to improve modern phrase-based SMT
systems
\cite{carpuatpsd,carpuat-wu:2007:EMNLP-CoNLL2007,carpuat2008evaluation}. In
Carpuat's work, classifiers are used to label multi-word expressions (phrases,
in the phrase-based SMT sense) with target language phrases. She demonstrates
how this is more appropriate in an an SMT setting than simply labeling
individual words with WordNet synsets, as had previously been attempted, and
showed significant improvements on a Chinese-English translation task.

There has also been work on using discriminative MaxEnt models to adapt
the ``forward" translation model of an SMT system, using richer
source-language context features \cite{vzabokrtsky-popel-marevcek:2010:WMT}.
Somewhat interestingly, while it has the same effect, this work does not
describe itself in terms of word sense disambiguation.

%% text from the HyTra paper
Although they did not present a complete MT system, there has also been work
on using WSD techniques for translation into lower-resourced languages, such as
the English-Slovene language pair, as in
\cite{vintar-fivser-vrvsvcaj:2012:ESIRMT-HyTra2012}. 

%%The Apertium team has a particular practical interest in improving lexical
%%selection in RBMT; they recently have been developing
%%a new system, described in \cite{tyers-fst}, that learns finite-state
%%transducers for lexical selection from the available parallel corpora. It is
%%intended to be both very fast, for use in practical translation systems, and
%%to produce lexical selection rules that are understandable and modifiable by
%%humans.

Francis Tyers, in his dissertation work \cite{tyers-thesis},
provides an overview of lexical selection systems and describes methods for
learning lexical selection rules based on available parallel corpora.
These rules make reference to the lexical items and parts of speech surrounding
the word to be translated. Once learned, these rules are intended to be
understandable and modifiable by human language experts. For practical use in
the Apertium machine translation system, they are compiled to finite-state
transducers.

%% XXX: maybe we should take all the stuff about Guampa out.
\subsection{Creating Corpora Through Crowdsourcing}
Many groups have addressed the problem of creating multilingual corpora through
crowdsourcing, in several variations. Some projects have explicitly had the
goal of creating sentence-aligned bitexts useful for training MT systems, while
others have aimed to create useful resources for humans. Also the means of
crowdsourcing has varied; some contributors have been paid, and others are
volunteers.

\subsection{Crowdsourcing Bitext for MT}
Vamshi Ambati's recent work \cite{ambati_naacl,ambati_act} focuses on using
crowdsourcing and active learning to produce a corpus for English-Spanish SMT.
His system has a representation of which words and phrases it should learn
about, finding n-grams that are common in a development set but have little
support in the training set, and then automatically creates Mechanical Turk
tasks to elicit translations containing translations of these n-grams. This
work makes use of the relatively large population of Internet users (and thus
MTurk users) familiar with both English and Spanish, and succeeded in building
a large bitext corpus quickly and cheaply.

The Joshua team at Johns Hopkins has also successfully used Mechanical Turk to
crowdsource the creation of corpora for SMT
\cite{post-callisonburch-osborne:2012:WMT}. In this work, the team built
large parallel corpora for many languages of the Indian
subcontinent\footnote{\url{http://joshua-decoder.org/indian-parallel-corpora/}}
and developed a simple approach for discovering the high-quality contributions,
in which MTurk users vote on which translations they consider the best.

%% DuoLingo \footnote{\url{http://duolingo.com}}, a company started by Luis von
%% Ahn, offers free language lessons to users 
%% XXX
%% There's also MonoTrans
%% \url{http://www.cs.umd.edu/hcil/monotrans/}
%% "... an iterative protocol in which monolingual human participants work
%% together to improve imperfect machine translations." 
%% Tradubi
%%Tradubi project
%%\url{http://wiki.apertium.org/wiki/Tradubi}
%%(project seems to have died, though)
%%then there's Amara and obviously DuoLingo...
%% OPUS
%% OPUS, "the open parallel corpus".
%% \url{http://opus.lingfil.uu.se/}

\subsection{Crowdsourcing Resources for People}

There are also a number of projects for building translations of documents with
human readers as the intended audience.
While the familiar DuoLingo \footnote{\url{http://duolingo.com}} is a business
that offers free online language lessons to learners and charges its customers
for having students translate documents, many of these other projects are
non-profit, community-driven, and cover less commonly taught languages.

Tatoeba \footnote{\url{http://tatoeba.org/}} (which is Japanese for ``for
example") is a repository of simple example sentences, with their translations
into other languages. Users may contribute new sentences or translations of
existing sentences into other languages. Interested users may also download the
entire Creative-Commons licensed database of sentences
\footnote{\url{http://tatoeba.org/eng/downloads}}, which includes text in a
large variety of languages, including some Guarani. As of this writing, there
are 474 Guarani sentences in Tatoeba.

In contrast with Tatoeba, Traduwiki \footnote{\url{http://traduwiki.org}} is a
project for building translations of documents rather than individual example
sentences. Users may upload new documents to translate, with the intention that
their writing reach a broader audience than it would reach in the original
language, and volunteers can translate individual segments. Unfortunately, this
project seems to have gone largely dormant, and the site is gradually
collecting spam edits. Unlike Tatoeba, the Traduwiki software does not seem to
be open source, and there is no obvious data export functionality -- Guampa,
our new software for online bilingual corpus construction, will fix both of
these problems.

\section{Online Language Tools for Guarani}
There are currently very few online language tools for Guarani; some of the
best available ones are iGuarani \footnote{\url{http://iguarani.com/}}, a
searchable online dictionary and gisting translation system developed by young
Paraguayan programmer Diego Alejandro Gavilán.

There is also a small searchable online dictionary developed by Wolf Lustig
\footnote{\url{http://www.uni-mainz.de/cgi-bin/guarani2/dictionary.pl}},
which has versions in Spanish, German, and English. He has graciously made this
dictionary available for our use, and this will provide a starting point for
our Spanish-Guarani translation system.

The Guarani activist and education community has a presence on the Web,
although a startlingly large fraction of this presence is a single author,
Dr. David Galeano Olivera, the president of the \emph{Ateneo de Lengua y
Cultura Guaraní}. He has collected a number of resources both about and in the
Guarani language
\footnote{\url{http://cafehistoria.ning.com/profiles/blogs/la-lengua-guarani-o-avanee-en}},
some in Spanish and some in Guarani, and will likely continue to do so.
