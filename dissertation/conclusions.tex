\chapter{Summary and Future Work}
\label{chap:conclusions}

In this work, we have described some approaches and running software for
cross-lingual word sense disambiguation, particularly in the setting where we
aim to translate from a resource-rich language into an under-resourced one, and
shown how they can be applicable for lexical selection when translating into
two different indigenous languages of South America, Guarani and Quechua.

We have extended our initial approach with the use of a number of techniques
that make use of the resources, both textual and software, available for our
source languages, including bitext corpora that pair our source language with
languages other than our intended target language. Finally, we have
demonstrated prototypes for how to integrate these CL-WSD classifiers into
existing machine translation software.

We have seen seen promising initial results with these approaches, as described
in XXX and our papers mentioned in that section.

%% XXX working here

\section{Future Work}

\section{Applying techniques like this to neural machine translation}
Since the start of this work, the field of machine translation has undergone a
dramatic shift, in which MT research has moved almost entirely from
phrase-based and tree-based SMT systems to models based on neural networks.

However, many of the difficulties addressed in this work still apply in the
Neural Machine Translation (NMT) setting. It would be reasonable to consider
NMT systems to be performing CL-WSD, in the sense that they have a
representation of the whole input sentence available while they are making
output decisions. This vector representation has been learned jointly with the
task of translating into the target language, so this process encapsulates much
of the feature engineering work that would go into building a CL-WSD system.
Thus it may not be sensible to add explicit CL-WSD classifiers to a neural MT
system, although this is an empirical question.

This leaves us with the issue, however, of how to learn from the available
corpora and tools that we have on hand when translating out of a resource-rich
language. How can we make use of our NLP tools for well-supported European
languages when translating into under-resourced languages?

In this work, we did not find it overwhelmingly helpful to use neural
embeddings based on relatively large corpora, at least not in a CL-WSD setting.
But can these monolingual source-language resources be used for NMT?

There are a number of open questions in neural machine translation; there do
not seem to be widely-accepted approaches for leveraging existing resources to
improve translation into under-resourced languages. ... %% XXX working here


%% probably just remove this section
%% \section{Morphological Generation}
%% %% future work, necessary for a complete MT system into morphologically rich
%% %% indigenous languages 

%% This can probably just be moved into a future work section.
%% We will also need to address both multi-word expressions and morphology and
%% how they interact with the CL-WSD system.  Thus far, we have assumed
%% one-to-many alignments, and labeled each source word with zero or more
%% lemmatized target words or word stems.  However in practice, we will want to
%% make use of multi-word expressions, and our translation systems will need to
%% generate appropriately inflected target language text.


\section{Acquiring larger bitext corpora}
\label{sec:crowdsourcing}




%% XXX this needs to be reworked for sure
As part of the ongoing work for the practical goal of building a useful
Spanish-Guarani MT system, we would like to build larger training corpora,
containing both bitext and monolingual Guarani text.
To help in the collection, we plan to build two websites:
a collaborative online space for building translations of documents, 
and a searchable repository of Guarani and bilingual documents.
Initial designs for both of these sites were done as a master's project in HCI
by Alberto Samaniego\footnote{\url{http://albsama.com}}, who will hopefully
continue collaborating on this project from his native Paraguay.

\section{Collecting Guarani Documents}
The first website we will develop is called ``Tahekami", which means
\emph{let's search together} in Guarani.
Tahekami is a repository of Guarani and bilingual documents that will allow
searching, browsing documents by tag, and uploading new documents.

An initial version of this site is already well underway
\footnote{\url{http://github.com/hltdi/gn-documents}}.  We have a working
search engine based on the Whoosh library
\footnote{\url{http://bitbucket.org/mchaput/whoosh/wiki/Home}} and some sample
documents -- twelve masters theses from the \emph{Ateneo}. We will need to
develop policies for which documents are permissible for distribution through
this site and work on integrating morphological analysis into the search
engine. Currently, new documents must be approved by an administrator before
being added to the index.

\section{Collecting Translations}
The second website, tentatively called ``Guampa"
\footnote{A ``guampa", in Paraguay, is the cup from which one drinks yerba mate
or tereré. The term ``guampa" is also local to Paraguay; in other parts of
South America, the container itself is called a ``mate".},
will be used by Guarani speakers and learners to produce translations of
relevant documents from Spanish to Guarani or vice-versa.
It will be something like a bilingual wiki, although the interface will
encourage users to edit sentences individually.
The software will segment the sentences in the initial
source-language documents and allow users to contribute translations for each
source sentence in turn, while showing the complete document context.
As a result of this, not only will will we be able to collect bitext training
data, but we will also produce useful translations.

Initially, this site will be seeded with documents from the Spanish and Guarani
Wikipedias. Successful translations of the Spanish-language articles could be
fed back into the Guarani Wikipedia. Other documents will be added by
Guarani-language educators and perhaps also pulled from Tahekami. Translations
may be assigned as homework by Guarani-language teachers.

The website will keep track of translations contributed by individual users;
there may be game-like features and community voting, where large number of
translations, or particularly good ones, are recognized, perhaps with virtual
prizes and badges.
Ideally, community management will be addressed by Paraguayan volunteers and
the gamification features can be built by contributors from the open source
world; this website and its richer features are not the primary focus of this
dissertation.

We may eventually collect enough bitext with this website such that it makes
sense to develop approaches for determining which sentences are the most
reliable and the most useful for training; this may correlate with quality
judgements from the human volunteers.
Investigating this relationship would make a good research question.

In the medium-term, this website will get an integrated ability to search
a translation memory and automatic suggestions from a machine translation
system\footnote{Features described in a presentation in Spanish here:
\url{http://www.cs.indiana.edu/~gasser/Taller2013/} ; English-language similar
presentation: \url{http://tinyurl.com/alexr-clingding-guarani} }
. While these features will be both useful and present a number of
interesting research questions, they are outside the scope of this
dissertation.
