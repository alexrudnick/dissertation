\chapter{Conclusions}
\label{chap:conclusions}
Thus far, we have prototyped our CL-WSD systems that make use of multilingual
evidence and sequence labeling techniques and seen promising initial results
with these approaches, as described in XXX and our papers
mentioned in that section.

Initial experiments are underway with building Tereré on top of the cdec
toolkit; we can already train Guarani-language LMs with KenLM
\cite{Heafield-estimate}, trained on the Guarani-language Wikipedia.
Additionally, we have mocked up the inclusion of CL-WSD features into the
decoder and started experimenting with writing SCFG transfer rules.

The Tahekami website is already well underway, and we have a web server on
which we can run Tahekami and Guampa. Fairly enthusiastic volunteers, including
developers from Paraguay, will likely continue helping in their development.

As a result of this work, we will have developed some new approaches for
CL-WSD and for building translation systems that target lower-resourced
languages. In this setting, phrase-based SMT is not feasible, so we have to
make effective use of our resources for the source language.
Practically, we should also have a new MT system for the Spanish-Guarani
language pair where there was previously none, an open-source reusable
package for helping RBMT systems make appropriate lexical choices, and a freely
available bitext corpus for a resource-poor language.

\section{Future Work}

\section{Morphological Generation}
%% future work, necessary for a complete MT system into morphologically rich
%% indigenous languages 
In the second pass, we will predict the appropriate morphological features will
with a discriminative sequence-labeling approach based on work at Microsoft
Research \cite{toutanova-suzuki-ruopp:2008:ACLMain}.
Thus both the transfer rules and the language model will be in terms of stemmed
Guarani.
As an alternative, we could adapt the techniques in
\cite{chahuneau:2013:emnlp} to generate translation rules that contain the
appropriately inflected target forms, just before running the decoder.
Rule-based approaches may also be sensible for generating Guarani morphology,
in some cases, and these will have to be investigated. In any case, once the
appropriate morphological features have been predicted, surface forms of
Guarani words will be generated with the FST-based morphological analyzer and
generator developed by Michael Gasser and described in
\cite{rudnick-gasser:2013:HyTra}.

%% This can probably just be moved into a future work section.
%% We will also need to address both multi-word expressions and morphology and
%% how they interact with the CL-WSD system.  Thus far, we have assumed
%% one-to-many alignments, and labeled each source word with zero or more
%% lemmatized target words or word stems.  However in practice, we will want to
%% make use of multi-word expressions, and our translation systems will need to
%% generate appropriately inflected target language text.


\section{Acquiring larger bitext corpora}
\label{sec:crowdsourcing}
As part of the ongoing work for the practical goal of building a useful
Spanish-Guarani MT system, we would like to build larger training corpora,
containing both bitext and monolingual Guarani text.
To help in the collection, we plan to build two websites:
a collaborative online space for building translations of documents, 
and a searchable repository of Guarani and bilingual documents.
Initial designs for both of these sites were done as a master's project in HCI
by Alberto Samaniego\footnote{\url{http://albsama.com}}, who will hopefully
continue collaborating on this project from his native Paraguay.
We have also had helpful software contributions from
Rodrigo Villalba Zayas\footnote{\url{http://github.com/rodrigovz}} --
also from Paraguay -- and more potential collaborators have expressed
interest in helping, from Paraguay, Indiana, and the broader open-source world.

\section{Collecting Guarani Documents}
The first website we will develop is called ``Tahekami", which means
\emph{let's search together} in Guarani.
Tahekami is a repository of Guarani and bilingual documents that will allow
searching, browsing documents by tag, and uploading new documents.

An initial version of this site is already well underway
\footnote{\url{http://github.com/hltdi/gn-documents}}.  We have a working
search engine based on the Whoosh library
\footnote{\url{http://bitbucket.org/mchaput/whoosh/wiki/Home}} and some sample
documents -- twelve masters theses from the \emph{Ateneo}. We will need to
develop policies for which documents are permissible for distribution through
this site and work on integrating morphological analysis into the search
engine. Currently, new documents must be approved by an administrator before
being added to the index.

\section{Collecting Translations}
The second website, tentatively called ``Guampa"
\footnote{A ``guampa", in Paraguay, is the cup from which one drinks yerba mate
or tereré. The term ``guampa" is also local to Paraguay; in other parts of
South America, the container itself is called a ``mate".},
will be used by Guarani speakers and learners to produce translations of
relevant documents from Spanish to Guarani or vice-versa.
It will be something like a bilingual wiki, although the interface will
encourage users to edit sentences individually.
The software will segment the sentences in the initial
source-language documents and allow users to contribute translations for each
source sentence in turn, while showing the complete document context.
As a result of this, not only will will we be able to collect bitext training
data, but we will also produce useful translations.

Initially, this site will be seeded with documents from the Spanish and Guarani
Wikipedias. Successful translations of the Spanish-language articles could be
fed back into the Guarani Wikipedia. Other documents will be added by
Guarani-language educators and perhaps also pulled from Tahekami. Translations
may be assigned as homework by Guarani-language teachers.

The website will keep track of translations contributed by individual users;
there may be game-like features and community voting, where large number of
translations, or particularly good ones, are recognized, perhaps with virtual
prizes and badges.
Ideally, community management will be addressed by Paraguayan volunteers and
the gamification features can be built by contributors from the open source
world; this website and its richer features are not the primary focus of this
dissertation.

We may eventually collect enough bitext with this website such that it makes
sense to develop approaches for determining which sentences are the most
reliable and the most useful for training; this may correlate with quality
judgements from the human volunteers.
Investigating this relationship would make a good research question.

In the medium-term, this website will get an integrated ability to search
a translation memory and automatic suggestions from a machine translation
system\footnote{Features described in a presentation in Spanish here:
\url{http://www.cs.indiana.edu/~gasser/Taller2013/} ; English-language similar
presentation: \url{http://tinyurl.com/alexr-clingding-guarani} }
. While these features will be both useful and present a number of
interesting research questions, they are outside the scope of this
dissertation.
