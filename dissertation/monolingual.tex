\chapter{Learning from Monolingual Data}
\label{chap:monolingual}
While in this work our target languages are under-resourced, we have many
resources available for the source languages. We would like to use these to
make better sense of the input text, giving our classifiers clearer signals and
better representations for lexical selection in the target language. The
approaches considered in this chapter make additional features or different
representations available to the CL-WSD classifiers based on knowledge of the
source language, either gleaned through unsupervised methods or baked into
extant tools. Since we have relatively little bitext available for
Spanish-Guarani and Spanish-Quechua, we will need to lean more on our Spanish
resources, software and data, in order to make better sense of the input text.

Perhaps most saliently for Spanish, we have abundant monolingual text
available, which suggests that we could use unsupervised methods to discover
regularities in the language, yielding better features for our classifiers.
This approach has been broadly successful in the literature
\cite{turian-ratinov-bengio:2010:ACL,baroni2014don}, and here we adapt some of
the methods explored in previous work on text classification to our task.

Concretely, in this chapter we explore labels learned from existing NLP tools
such as off-the-shelf taggers and parsers for Spanish, Brown clustering
\cite{brown1992class}, and two related approaches to neural embeddings. There
are of course other related methods that one could investigate, especially
making use of the broader literature on distributional semantics, but these
will be left to future work. First we will describe the methods used in some
detail, and then towards the end of the chapter, in Section
\ref{sec:monolingual-experiments}, we describe experiments and present
experimental results.

\section{Monolingual features from other NLP tools}
There are a large number off-the-shelf NLP tools available for Spanish; here we
will look into POS taggers and syntactic parsers specifically. Tagging can help
us capture abstractions over particular word types, and provides some
disambiguation on its own\footnote{For example, perhaps a noun in the window
surrounding a focus word is indicative of a particular meaning. More
concretely, we could imagine the CL-WSD system benefitting from distinguishing
the Spanish \emph{poder} as the infinitive verb ``to be able" from the noun
interpretation ``power, ability".}. Similarly, syntactic structure may also
provide useful features; a verb may have a drastically different translation
in the target language based on its objects.

In the Chipa software, we can make use of arbitrary annotations for the input
text (see Section \ref{sec:annotations}), so adding more features based on
analysis by external tools is straightforward.

As a first step, we synthesize features based on the part-of-speech tags of the
tokens in a window around the current focus word, and using a syntactic parser,
the heads and children of the current focus word. In principle, we could use
other annotations, such as sense annotations from a monolingual WSD system,
given a good one. This idea is akin to one that we will explore in Chapter
\ref{chap:multilingual}, in which we use Chipa itself (trained for other
language pairs, and with larger data sets) to provide annotations, effectively
using some other target language as a tagset for WSD.

For POS tagging, we run the open source FreeLing text analysis suite
\cite{padro12} on input sentences. FreeLing can perform a number of analyses
for Spanish, including POS tagging, dependency parsing, lemmatization, and
named entity recognition, of which the latter two are part of the standard
preprocessing done for all experiments in this work. When all the text for an
experiment is known beforehand, as in the experiments reported in this chapter,
we can run FreeLing during the data annotation step (see Section
\ref{sec:datasetsandpreprocessing}) and simply record FreeLing's output as
additional features. When running on novel test sentences, as in server mode,
we must run it on those sentences just before inference time.

Similarly, for syntactic features, we use MaltParser\cite{Nivre06maltparser:a}
\footnote{Available at \url{http://maltparser.org/} ; in this work we use
version 1.9.0 of the parser, and the ``espmalt" pretrained model, which is
available at \url{http://www.iula.upf.edu/recurs01_mpars_uk.htm}, and was
trained on the IULA Treebank\cite{MARIMON12.519}, by researchers from the IULA
group at Universitat Pompeu Fabra.} to get dependency parses of the input
sentence. This is also performed as a corpus annotation, making the syntactic
relationships for each token available during feature extraction. The parser
here depends on the POS tags produced by FreeLing. Thankfully, they use the
same tag set, but as with any pipeline of NLP systems, using the inferred
output from the tagger as input to the parser carries the risk that errors at
early stages in the pipeline could propagate and cause problems at the later
stages of processing\footnote{Here we note this concern but move on; it is a
very general problem, and solutions to it are an open area of research.}.  The
parser also depends on the coarse-grained ``universal" POS tags
\cite{PETROV12.274}, which we manually map from the IULA tagset to universal
coarse tags.

\begin{figure*}
  \begin{centering}
  \begin{tabular}{|r|p{11cm}|}
    \hline
    name          & description  \\
    \hline
    \texttt{postag}    & part-of-speech tag for the focus token \\
    \hline
    \texttt{postag\_left}  & POS tag for the token to the left of focus token \\
    \hline
    \texttt{postag\_right} & \emph{ibid.}, for the right \\
    \hline
    \texttt{head\_lemma} & lemma of the focus word's syntactic head, or ROOT if
    it has no head). \\
    \hline
    \texttt{head\_surface} & \emph{ibid.}, but for the syntactic head's surface
    form. \\
    \hline
    \texttt{child\_lemma} & lemma of the syntactic child or children of the
    focus word. Feature appears multiple times for multiple children. \\
    \hline
    \texttt{child\_surface} & \emph{ibid.}, but for the children's surface
    forms. \\
    \hline
  \end{tabular}
  \end{centering}
  \caption{Additional syntactic features}
  \label{fig:syntacticfeatures}
\end{figure*}

\section{Brown Clustering}
The Brown clustering algorithm\cite{brown1992class}, also known as IBM
clustering, as it was developed by the Candide group at IBM, takes an
unannotated text corpus as input and assigns each word type found the corpus
into hierarchical clusters such that types in the same cluster have similar
usage patterns, according to the bigram statistics of the corpus. The tree of
clusters is binary-branching, so the name of a cluster is simply its path from
the root of the tree.  The desired number of ``leaf" clusters must be set ahead
of time, as a tunable parameter.

We can use this clustering to create more features for our classifiers,
considering the clusters into which a word type falls as a tag for instances of
that word. These annotations describe a more abstract, less sparse
representation than surface forms or lemmas, hopefully providing useful
semantic and syntactic generalizations.

Brown clustering uses greedy optimization to find cluster assignments. Each of
the word types present in the input corpus must be placed into one of a fixed
number of hierarchical clusters, such that the assignment attempts to maximize
the probability of the input corpus.  As the original intention of the approach
was for class-based language models, the scoring function is the mutual
information between two immediately subsequent tokens (i.e., a bigram).
Concretely, the optimization process is searching for a clustering $C$ that
maximizes the probability of the corpus $\boldsymbol{w}$, according to the
formula in Figure \ref{fig:brownprob}.

\begin{figure*}

  \begin{equation} \label{eq:brownclassprob}
  P(\boldsymbol{w}; C) = \prod_{w_i} p(w_i | C(w_i)) p(C(w_i) | C(w_{i-1}))
  \end{equation}

  \caption{The Brown clustering expression for the probability of a corpus with
  a specific clustering $C$. It is the product, for each token, of the
  probability of that token given its cluster, and the probability of that
  current cluster given the previous cluster. This is analogous to the
  ``emission" and ``transition" probabilities used in an HMM-based tagger.}
  \label{fig:brownprob}
\end{figure*}

Finding the globally optimal assignment turns out to be an intractable problem,
but several greedy approaches that find local optima have been explored in the
literature.  Notably, in addition to Liang's approach, Franz Och's
\texttt{mkcls} package (familiar to Moses users, and described in
\cite{och1999efficient}) optimizes the same function. In any case, with the
available software running on modern hardware, we can find a clustering for the
corpora used in this work in a fairly short time -- a few hours at most.

In applying Brown clustering to this task, we would like to answer several
questions.

\begin{itemize}
  \item Can learning Brown clusters from  monolingual source-text resources 
    improve our performance on this CL-WSD task?
  \item Does more source-language text help us learn more helpful Brown
  clusters, with respect to the CL-WSD task?
  \item How does the genre of the input text affect our performance, and how
  does this relate with the size of the text? Is there a size at which we
  can learn a better clustering that overcomes genre mismatches?
  \item What kinds of preprocessing should we do on the source text? Most
  saliently, should the source text be lemmatized? We might expect that Brown
  clustering will help us find abstractions over syntax, but perhaps
  lemmatization would help it find more semantic abstractions in practice. Or
  perhaps it would be more effective to use the same input corpus in both
  lemmatized and inflected forms.
\end{itemize}

We can use such a clustering $C$ (i.e., a mapping from word types to clusters)
to extract a number of different kinds of features for our classifiers; we may
also choose a variety of inputs to the clustering algorithm in the first place.

%% Some of the features ...

We tried, for example, dropping a bag of all the clusters that occur in the
sentence, and prefix features for all the clusters that appear too!

It might be the case that we really just need to add the very local context --
three words on a side? In Turian et al, they do two words on a side.

\begin{figure*}
  \begin{centering}
  \begin{tabular}{|r|p{11cm}|}
    \hline
    name          & description  \\
    \hline
    \texttt{brown\_bag}    & Bag of all Brown clusters for the entire sentence \\
    \hline
    \texttt{brown\_window}  & All Brown clusters for a window around the focus word\\
    \hline
    \texttt{flat\_brown\_bag} & \emph{ibid.}, ... \\
    \hline
    \texttt{flat\_brown\_window} & \emph{ibid.}, ... \\
    %% XXX: add some more
    \hline
  \end{tabular}
  \end{centering}
  \caption{Features extracted from Brown clusters}
  \label{fig:brownfeatures}
\end{figure*}


\section{Brown clustering in practice}
In this work we use Percy Liang's implementation \footnote{Available at
\url{https://github.com/percyliang/brown-cluster}} of Brown clustering
\cite{Liang05semi-supervisedlearning}.
We have run the tool on several corpora, varying in
size from the Bible (roughly 30K sentences), to the rather larger Europarl
corpus \cite{europarl} (2 million sentences), to larger still, a dump of the
Spanish-language Wikipedia (20 million sentences).
%% XXX: working here

%% TODO: include examples of the clusters that we learn from different corpora
%% specify lemmatized versus not lemmatized

Here's an interesting thing to note: when we do clustering on lemmas, we seem
to pick up more on ``semantic" categories of things. Note the ``infrastructure"
cluster below -- it's got words that all pertain to infrastructure!

When we don't use lemmas, the cluster that contains ``infraestructura" contains
these words: \emph{formación educación tecnología infraestructura publicidad
enseñanza ocupación religión vivienda ética propaganda banda fusión huelga
creatividad medicina manipulación tribuna cualificación comida}. It seems like
this is more about grammatical gender: those are all singular feminine nouns.

In fact, the words in many of the clusters all seem to have the same
grammatical gender and number. To get more semantic similarity, maybe we should
cluster on lemmas -- otherwise the morphological features of Spanish could
overwhelm the clustering approach.

\TODO{replace with current example}
\begin{figure*}[t!]
  \begin{tabular}{|r|p{10cm}|}
    \hline
    category  & top twenty word types by frequency \\
    \hline
    countries & francia irlanda alemania grecia italia españa rumanía portugal polonia suecia bulgaria austria finlandia hungría bélgica japón gran\_bretaña dinamarca luxemburgo bosnia \\
    \hline
    more places & kosovo internet bruselas áfrica iraq lisboa chipre afganistán estrasburgo oriente\_próximo copenhague asia chechenia gaza oriente\_medio birmania londres irlanda\_del\_norte berlín barcelona \\
    \hline
    mostly people & hombre periodista jefes\_de\_estado individuo profesor soldado abogado delincuente demócrata dictador iglesia alumno adolescente perro chico economista gato jurista caballero bebé \\
    \hline
    infrastructure & infraestructura vehículo buque servicio\_público cultivo edificio barco negocio motor avión monopolio planta ruta coche libro aparato tren billete actividad\_económica camión \\
    \hline
    common verbs & pagar comprar vender explotar practicar soportar exportar comer consumir suministrar sacrificar fabricar gobernar comercializar cultivar fumar capturar almacenar curar beber \\
    \hline
  \end{tabular}
\caption{Some illustrative clusters found by the Brown clustering algorithm on
the Spanish Europarl data. These are five out of $C=1000$ clusters, and
were picked and labeled by hand. The words listed are the
top twenty terms from that cluster, by frequency.}
\label{fig:clusters}
\end{figure*}

Figure \ref{fig:clusters} shows some illustrative examples of clusters that
we found in the Spanish Europarl corpus.  Examining the output of the
clustering algorithm, we see some intuitively satisfying results; there are
clusters corresponding to the names of many countries, some nouns referring to
people, and common transitive verbs. Note that the clustering is unsupervised,
and the labels given are not produced by the algorithm.

We additionally show the effects on classification accuracy of adding features
derived from Brown clusters, with clusters extracted from both the Europarl
corpus and the Spanish side of our training data.
We tried several different
settings for the number of clusters, ranging from $C=100$ to $C=2000$.
In all of our experimental settings, the addition of Brown cluster features
substantially improved classification accuracy. We note a consistent upward
trend in performance as we increase the number of clusters, allowing the
clustering algorithm to learn finer-grained distinctions.
The training algorithm takes time quadratic in the number of clusters,
which becomes prohibitive fairly quickly, so even finer-grained distinctions
may be helpful, but will be left to future work. On a modern Linux
workstation, clustering Europarl (~2M sentences) into 2000 clusters took
roughly a day.

The classifiers using clusters extracted from the Spanish side of our bitext
consistently outperformed those learned from the Europarl corpus. We had an
intuition that the much larger corpus (nearly two million sentences) would
help, but the clusters learned in-domain, largely from the Bible, reflect
usage distinctions in that domain. Here we are in fact cheating slightly, as
information from the complete corpus is used to classify parts of that corpus.


\section{Neural Word Embeddings}
Another rich source of features that has proved useful for many text
classification problems in recent years is neural word embeddings, perhaps most
famously developed in the work of Mikolov et al. \cite{mikolovword2vec} and
their associated open source implementation of the technique,
word2vec\footnote{Available at
\url{https://code.google.com/archive/p/word2vec/}}. In this work we investigate
the use of ``word2vec"-style word embeddings, as well as the related technique,
``doc2vec" or ``paragraph vectors"
\cite{dai-document-embedding-2015,quocle-distributed-representations-2014}.
These techniques let our classifiers operate on lower-dimensional dense vectors
(of a few hundred dimensions, typically), as opposed to the high-dimensional
and sparse vectors typically used in earlier NLP literature\footnote{
What we might consider to be ``symbolic" binary features used in machine
learning systems for NLP in recent decades -- such as ``the word `dog` appears
in the sentence" -- are effectively equivalent to these very sparse ``one hot"
representations of words, in which, for a vocabulary of size $|V|$, words are
represented by a length-$V$ vector in which all but one of the elements are 0}.

There is a rich literature on continuous representations for words; the idea
has a long lineage in NLP, and we could try any number of dimensionality
reduction or distributional semantics approaches here. Recent empirical work by
Baroni et al \cite{baroni2014don} (summed up by the title of the paper, ``Don't
Count, Predict!") has shown that representations built during discriminative
learning tasks are typically more effective for text classification problems
similar to ours.

Unlike Brown Clustering, which infers hierarchical clusters for each word type,
but like other embedding techniques, word2vec learns multidimensional
representations for word types, turning the very sparse ``one hot"
representation in which a words with similar uses or meanings do not have any
obvious similarity into a much denser continuous space in which words with
related meanings are placed nearby. These ``embeddings", or placements of word
types into a continuous space, are learned during some other classification
task, and are performed by the early layers of a neural network.

The word2vec model in particular has two variations, useful in different
contexts. One, called Continuous Bag-of-Words (CBOW) learns a classifier that
can predict individual words based on their context, while the ``skip-grams"
variant does the reverse and learns to predict context words based on an
individual focus word. The received wisdom\footnote{According to the TensorFlow
tutorial on word2vec, available at
\url{https://www.tensorflow.org/tutorials/word2vec}} is that CBOW is more
appropriate when training on smaller data sets, but this is an empirical
question and here we try out both variants.

In either case, running a word2vec training process results in a static mapping
from word types to their embeddings in a vector space of a given size,
typically a few hundred dimensions. These embeddings have been shown to be
helpful as features in a number of different NLP tasks \cite{baroni2014don},
allowing us to learn richer representations of word types from plentiful
unannotated monolingual text. This effectively turns what was a purely
supervised task, requiring labeled training data, into a semi-supervised task,
in which the representation-learning phase can be carried out on unlabeled data
by synthesizing a supervised learning task that can be performed with simple
monolingual text as its training data.

We trained a variety of word2vec embeddings for our text classification tasks,
training on the Spanish-language section of the Europarl corpus (roughly 57
million words and 2 million sentences) and a dump of Spanish Wikipedia (449
million words, 20 million sentences). We learned embeddings using both CBOW and
skip-gram approaches, in 50, 100, 200, and 400 dimensions. In all cases, we
used the associated \texttt{word2phrase} tool, distributed with the initial
\texttt{word2vec} implementation, which finds collocations in the input text
and treats them as individual tokens. The same collocations must be identified
in the input text for their embeddings to be used. At lookup time, we keep a
list of all the multiword expressions present in the dictionary of embeddings,
sorted from longest to shortest, first considering number of tokens, then
breaking ties by considering number of characters, and greedily replace any of
these multiword expressions (starting with the longest) found in the input text
with the token representing that MWE.

\subsection{From word embeddings to classification features}
We should note that the embeddings learned for a word type are fixed for that
word type, and do not reflect the context of a particular token, so a focus
word's embedding vector will not provide useful classification features on its
own. This leaves us with the problem of how to make use of the word vectors in
a given sentence. Concretely, we must decide how to combine several word
vectors together, and which word vectors in a sentence to choose for
combination. The typical approach (XXX citation) is to take element-wise sums
or averages over the embedding vectors for the tokens in a context.

There are a variety of approaches we could take, even within summing and
averageing. We could sum (element-wise) all of the word vectors for all the
tokens in the entire sentence. We could only consider the words in a narrow
window around the focus word. Or we could perform a weighted sum, in which
words nearer to the focus word are given more weight, but these weights drop
off according to some function, the farther away they are from the focus word.

%% XXX working here

%% XXX add some citations for papers where people use word embeddings as
%% features

\section{Neural Document Embeddings}
The same researchers that produced word2vec and their collaborators have also
developed approaches that build representations for sequences of text, rather
than individual word types, described in
\cite{dai-document-embedding-2015,quocle-distributed-representations-2014}.
This approach, called ``paragraph vectors" or ``doc2vec" takes unlabeled text
and learns representations for word types jointly with representations for each
\emph{document}. Here a ``document" is an arbitrary sequence of tokens, perhaps
a single sentence, or perhaps much longer, depending on the intended
application.

Similar to word2vec, here doc2vec uses a neural network with a single hidden
layer to predict words in the current context. Here however, the hidden layer
contains not only an embedding trained for word types, but also an embedding
for the current document. Training proceeds by loading the current
representation for the current word and the current document, attempting to
predict some other word in the context (analogous to training for word2vec),
and then updating both representations based on the gradients between the
desired output and the actual output. 

Also similar to word2vec, there are two variants of the approach that can be
used for training. One variant of doc2vec, the ``Distributed Memory Model of
Paragraph Vectors", uses a paragraph embedding combined with word embeddings
for several words in the current context to predict a single individual word.
The other variant, the ``Distributed Bag of Words model of Paragraph Vectors",
simply uses the current paragraph vector to predict all of the words in the
current context, which despite containing ``bag of words" in its name, is more
closely analogous to the skip-gram artitecture for word2vec
\cite{quocle-distributed-representations-2014}, since the model is trained to
predict the entire context based on a single embedding.

In either case, the resulting stored model is the set of embeddings for
individual word types. The embeddings for any particular document in the
training set are not stored, since that document is unlikely to appear again at
inference time, but the embeddings for individual word types provide a useful
generalization, since they constrain embeddings for new documents.

At inference time, doc2vec produces new vectors describing previously unseen
documents based on these fixed word embedding vectors. In our work, as in many
text classification tasks, documents will be input sentences, or perhaps
sentence-length Bible verses. To produce a representation for a new document,
the embeddings for word types are held constant and we run the optimization
process (stochastic gradient descent) to infer an embedding for the current
document, such that it helps predict words in the current context.

doc2vec thus provides a straightforward approach for learning to produce
representations of new documents, based on an unlabeled training corpus. Here
we use the implementation of doc2vec provided by the gensim package
\cite{rehurek-lrec}, with its default settings, which means that training is
carried out with the ``distributed memory" model.
During training data annotation, we take each source-language sentence and
infer a new embedding for it based on a doc2vec model trained on 20 million
sentences dumped from the Spanish-language wikipedia. These sentence-level
embeddings are stored as a token annotation on the first token of each
sentence, so they can be made available during feature extraction.

\section{Experiments}
\label{sec:monolingual-experiments}
Here we repeat the experiments from Chapter \ref{chap:evaluation}, with the
addition of features extracted from syntactic tools, Brown clusters, and neural
While there are many possible combinations, we do not try all of them
exhaustively.

\begin{itemize}
  \item es-gn, es-qu with pos tags
  \item es-gn, es-qu with syntactic heads
  \item es-gn, es-qu with syntactic children
  \item es-gn, es-qu with all syntactic features
\end{itemize}

\begin{itemize}
  \item es-gn, es-qu with europarl brown clusters
  \item es-gn, es-qu with wikipedia brown clusters
  \item es-gn, es-qu with europarl and wikipedia brown clusters
  \item es-gn, es-qu with word2vec embeddings of dimensions (50, 100, 200, 400)
  from both europarl and wikipedia
\end{itemize}

\begin{itemize}
  \item es-gn, es-qu with europarl embeddings
  \item es-gn, es-qu with wikipedia embeddings
  \item es-gn, es-qu with europarl and wikipedia embeddings
  \item es-gn, es-qu with pos tags and syntactic heads
\end{itemize}

\begin{itemize}
  \item combine the most promising ones ??
  %% XXX: we don't actually have code to combine word2vec features with the
  %% default features yet.
\end{itemize}
