\chapter{Learning from Multilingual Data}
\label{chap:multilingual}
As discussed in previous chapters, while our target languages are
under-resourced, we have many resources available for our source languages.
Concretely for Spanish, in addition to the abundant monolingual text and
off-the-shelf NLP tools, as discussed in Chapter \ref{chap:monolingual}, we
have a significant amount of bitext, pairing Spanish with languages other than
our under-resourced target languages. We would like to be able to learn from
these available bitext corpora when translating from Spanish to Guarani and
Quechua, and ideally in general from resource-rich source languages into
under-resourced target languages.

Each bitext corpus may contain useful examples of a given source language word,
and senses of that word may be lexicalized differently in the various target
languages, giving us clues about the meaning of each token in its context.
Selecting a contextually correct translation for source-language words is
evidence that we have understood the meaning of those words, at least
implicitly, in as far as sense distinctions are surfaced in the target
language. So we would like our system to be able to learn relationships between
the senses of a given source word, as they are represented in different target
languages. Two target languages may happen to surface similar sense
distinctions, perhaps due to being related languages, or perhaps because a word
sense ambiguity is unusual in the source language, or simply by coincidence.
Additionally, a combination of translations into several languages may provide
evidence for a certain lexical choice in the target language of interest.

In this chapter, we present strategies for learning from the available bitext
corpora that do not include our under-resourced target languages, to help us
make better CL-WSD decisions when translating into them. Primarily we discuss a
``classifier stacking" approach, in which we train CL-WSD classifiers
translating from Spanish to other European languages, and use the outputs of
these classifiers as features for Spanish to Quechua and Spanish to Guarani.

The approaches in this chapter are particularly informed by the work of Els
Lefever \emph{et. al} (see especially
\cite{lefever-hoste-decock:2011:ACL-HLT2011}), in which entire source sentences
are machine-translated into several different target languages just before
feature extraction, and these entire sentences are used to produce signals for
CL-WSD classifiers. One drawback of this technique is that it could be
considered unwieldy from a software engineering perspective; it requires
multiple complete MT systems to perform CL-WSD, which is rather complex when we
want to use CL-WSD as a subcomponent of a machine translation system in the
first place.

In earlier work, considering the work of Lefever \emph{et. al}, we developed
prototype CL-WSD systems that made use of multilingual evidence
\cite{rudnick-liu-gasser:2013:SemEval-2013} and produced some of the top
results in a SemEval shared task on CL-WSD \cite{task10}.
Our systems trained with multilingual evidence posted better performance than
the one that used monolingual features; our top results in every language came
from either classifier stacking or the classifier based on Markov networks,
which we will discuss briefly in Section~\ref{sec:multilingual-mrf}. This
suggests that it is possible to use evidence in several parallel corpora for
CL-WSD without translating source sentences into many target languages.

For the rest of the chapter, we will describe some of the bitext corpora that
we can use for Spanish, the Markov network approach described in our SemEval
entry, the classifier stacking approach, and present experiments using the
classifier stacking approach for Spanish-Guarani and Spanish-Quechua CL-WSD
tasks.

\section{Multilingual corpora} 

\subsection{Europarl}
There are quite a few bitext resources available for many European languages,
especially the official languages of the European Union. Through the Europarl
corpus \cite{europarl}, for example, we have bitext corpora in which Spanish is
paired with English, German, French, and 17 other European languages.

Using the automatic tools distributed with the corpus\footnote{Available
online at \url{http://statmt.org/europarl/}}, we produced sentence-aligned
bitext, ending up with roughly 1.7 sentence pairs per language pair, though
this number varies somewhat. After extracting these sentence pairs, we ran the
extracted bitext through our standard preprocessing pipeline.

%% XXX merge these two bits together
The bitext corpora are prepared with the tools distributed with the Europarl
corpus, which conveniently produce sentence-aligned and tokenized bitext.  We
then preprocess the bitext with the same steps we used on our Bible text, as
described in Section~\label{sec:datasetsandpreprocessing}. Where possible, we
lemmatize the target-language text with FreeLing.  However, as of this writing
FreeLing does not support Dutch, so for Dutch, we use the Frog text analysis
tool\footnote{Frog was developed by groups at Tilburg University and the
University of Antwerp. It is available at
\url{http://languagemachines.github.io/frog/}} \cite{tadpole2007}. With the
exception of the different lemmatizer for Dutch, the preprocessing steps are
analogous to those described in the previous chapters, resulting in
automatically aligned Spanish-Dutch, Spanish-English, Spanish-French,
Spanish-German, and Spanish-Italian bitext corpora.


%% XXX what else to say about Europarl?

\subsection{Bible translations}

For the additional European languages, translations were made available as part
of a project by Christodouloupoulos \emph{et
al.}\cite{Christodouloupoulos2015}\footnote{Their provided Bible text is
available online at \url{https://github.com/christos-c/bible-corpus}}

%% XXX Bible translations

\section{Considering domain mismatches}
In comparing the use of different corpora for our classifier stacking approach,
we want to ask to what extent the domain mismatch will play a role.
The subjets discussed in European Parliament may not match that of our
Spanish-Guarani and Spanish-Quechua corpora; furthermore, perhaps the Spanish
words used may not match at all.

Exploring the corpora a bit, we find that, indeed, many of the word types used
in our Spanish Bible translation either never appear, or only appear very
rarely in the Europarl text. And while there are many word types that are
relatively commonly used in the Bible text that never appear in our Europarl
corpus, these comprise a small fraction of the Bible text overall, and thus a
small number of the CL-WSD problem instances in our experiments.

%% XXX measurements go here

\section{CL-WSD with Markov Networks}
\label{sec:multilingual-mrf}
In our SemEval entry \cite{rudnick-liu-gasser:2013:SemEval-2013}, we
investigated a CL-WSD approach based on Markov networks (also known as ``Markov
Random Fields"), building a network of interacting variables (see Figure
\ref{fig:pentagram}) to solve CL-WSD classification problems for the five
target languages of the SemEval 2013 task \cite{task10}. In this task, we were
given English source sentences with an annotated focus word (from a given set
of possible focus words types), and asked to make correct lexical selections
for translating into Dutch, French, German, Italian and Spanish. The ground
truth for the task was determined by human annotators familiar with those
languages.

Here the nodes in our Markov network represent random variables that take on
values corresponding to the possible translations for each of the five target
languages. The probability distributions over these translations are produced
by language-specific maximum entropy classifiers, effectively applying the
baseline Chipa system on the given input sentence.

The edges in the graph correspond to pairwise potentials that are derived from
the joint probabilities of target language labels co-occurring in the available
bitext for the two target languages along that edge of the graph. This approach
thus requires a bitext corpus for each pair of languages in the set of
languages involved; for the SemEval task, we worked with a subset of the
Europarl corpus, provided by the task organizers, in which every sentence was
provided for all six languages.

We frame the task of finding the optimal translations into five languages
jointly as a MAP inference problem, wherein we try to maximize the joint
probability of all five variables, given the single source language sentence.
We perform inference with loopy belief propagation
\cite{DBLP:conf/uai/MurphyWJ99}, which is an approximate but tractable
inference algorithm that, while giving no guarantees, often produces good
solutions in practice.
We used the formulation for pairwise Markov networks that passes messages
directly between the nodes rather than first constructing a ``cluster graph",
which is described in \cite[\S 11.3.5.1]{Koller+Friedman:09} of Koller and
Friedman's book on graphical models. This Markov network approach has the
theoretically satisfying property that it takes seriously the uncertainty
present in the predictions of each of the component classifiers and solves the
entire problem jointly. 

Intuitively, at each time step loopy belief propagation passes messages around
the graph that inform each neighbor about the estimate, from the perspective of
the sender and what it has heard from its other neighbors, of the minimum
penalty that would be incurred if the recipient node were to take a given
label. As a concrete example, when the \emph{nl} node sends a message to the
\emph{fr} node at time step 10, this message is a table mapping from all
possible French translations of the current target word to their associated
penalty values. The message depends on three things: the probability
distribution from a monolingual classifier just for Dutch, joint probabilities
estimated from our Dutch-French bitext, and the messages from the \emph{es},
\emph{it} and \emph{de} nodes from time step 9.

\begin{figure}
  \begin{center}
  \includegraphics[width=5cm]{pentagram.pdf}
  \end{center}
  \caption{The network structure used in the MRF system for SemEval: a complete
  graph with five nodes, in which each node represents the random variable for
  the translation into a target language.}
  \label{fig:pentagram}
\end{figure}

\begin{figure}
  \begin{center}
  \includegraphics[width=5cm]{gn-qu-mrf.pdf}
  \end{center}
  \caption{Hypothetical network structure -- not fully connected -- that could
  be used for a Markov network translating from Spanish to all of the six
  languages shown, if we had a bitext corpus for Spanish and every
  other language, for also for German-Quechua.}
  \label{fig:gn-qu-mrf}
\end{figure}

While we were able to achieve fairly good results with these MRF-based
classifiers on the Semeval CL-WSD task, our strategy for setting the weights in
the Markov network requires, for each edge in the graph, a parallel corpus for
the corresponding language pair. The bitext for each language pair need not be
mutually parallel among all of the languages present; each edge in the graph
may correspond to an unrelated bitext corpus.  Also, in principle the graph
need not be fully connected, as it was for our SemEval entry; see
Figure~\ref{fig:gn-qu-mrf} for a hypothetical graph structure that could be
used with the right bitext corpora available. But this approach
does require bitext between the source language and all other languages
involved, as well as our target language of interest and at least one of the
other language, so that our choices for that other language can inform our
choices for the target language of interest.

Considering the limited bitext resources available, this approach is less
easily applicable for the under-resourced target language use case; concretely,
we do not have Europarl corpora available for Quechua and Guarani. In the
classifier stacking approach, it is clear how we can include information from
many heterogeneous sources. The Markov network approach may be useful for
future CL-WSD systems in other settings, but for now we will not make further
use of it for translating into under-resourced languages, since we do not have
many parallel corpora available for Guarani or Quechua.

\section{Classifier stacking}

The simpler approach that we use in practice in this work in a form of
classifier stacking. In order to predict the translation of a word into our
intended target language, we use features based on our predictions for
translations of tokens in the current sentence into \emph{other} available
target languages.
For example, for our SemEval prototype systems, in order to translate an
English word into Spanish, we predict that word's translations into French,
Italian, Dutch and German, and then encode those predicted translations as
features for our English to Spanish classifier.

This approach only requires that the classifiers used for generating new
features make \emph{some} prediction based on the input text. These classifiers 
need not be trained from the same source text, or depend on the same features,
or even necessarily output words as features. Any annotation on the text that
we believe is meaningful could be used for feature extraction; for example, we
could use this technique using a monolingual WSD system that output word sense
annotations.

\subsection{Annotating the bitext with stacking predictions}

Here we train the baseline Chipa system, with the baseline feature set, on both
Europarl bitext pairing Spanish with five other languages, and on verse-aligned
Bible translations, for the same five European target languages. Following
earlier work, including our system for SemEval and that of Lefever \emph{et
al.}, the target languages used are Dutch, English, French, German, and
Italian. For all of the stacking classifiers mapping from Spanish to another
European language, we used a random forest classifier and the ``regular"
setting.

%% XXX need a better segue
We annotated our bitext for Spanish-Guarani and Spanish-Quechua,
training classifiers for each of the in-vocabulary focus words for that
language pair (see Section \ref{sec:exploring}) based on the additional
provided bitext, and marking up each source-language word in the bitext with
its predicted translation into the available European languages.

\section{Experiments}
\label{sec:multilingual-experiments}

The experiments in this chapter are analogous to the ones in previous chapters,
with the addition of features based on classifier stacking. We ran the same
machine learning algorithms on the same training and test sentences as in
previous chapters, for both Spanish-Guarani and Spanish-Quechua.

For all tasks in this section, we trained with the baseline features described
in Chapter~\ref{chap:baseline}, as well as the new features introduced in this
chapter; these are presented in Figure~\ref{fig:stackingfeatures}.  We also
trained classifiers with the addition of the syntactic features described in
Chapter~\ref{chap:monolingual}. Classifier stacking experiments were done with
both ``English only" variants, in which the only additional features added were
based on the Spanish-English classifier, and the ``five languages" variant,
where we added features based on the predictions for Spanish to Dutch, English,
French German and Italian. The predictions provided were based on classifiers
trained on Europarl, on the Bible, and then both classifiers together; this
last approach added two separate sets of stacking features to the classifiers.

\begin{figure*}
  \begin{centering}
  \begin{tabular}{|p{3.5cm}|p{11cm}|}
    \hline
    name          & description  \\
    \hline
    \texttt{stacking\_en} & Predicted translation of the current token into
    English, if one was available. Feature is not present if no prediction was
    made for this token. \\
    \hline
    \texttt{stacking\_de}, \texttt{stacking\_fr}, \texttt{stacking\_it},
    \texttt{stacking\_nl} & \emph{ibid.}, but with a prediction into the
    corresponding language.\\
    \hline
    \texttt{stacking\_window} & The predictions for any tokens in the
    surrounding context window, into any of the available languages for this
    run. \\
    \hline
  \end{tabular}
  \end{centering}
  \caption{Classifier features based on classifier stacking, used in these
  experiments}
  \label{fig:stackingfeatures}
\end{figure*}

\section{Experimental Results}
\label{sec:multilingual-results}

This section contains several tables of numbers; as before, for each set of
experimental results presented, the top result for each setting is presented in
\emph{italics}, and the top result for a setting presented in the whole
chapter, or a result tied for the top result, is presented in \textbf{bold}.

\subsection{Results: classifier stacking with Europarl}

In Figure \ref{fig:europarl-stacking-results}, we present results for the
experiments with stacking using the Europarl bitext.

\begin{figure*}
  \begin{centering}
  \begin{tabulary}{\textwidth}{|R|L|L|L|L|}
    \hline
    classifier & es-gn regular & es-gn non-null & es-qu regular & es-qu non-null \\

    \hline
    MFS    & 0.456 & 0.498 & 0.435 & 0.391 \\
    \hline
    \hline

    \multicolumn{5}{|l|}{maxent l1} \\
    \hline
    baseline features & 0.461 & 0.506 & 0.444 & 0.414 \\
    \hline
    +all syntactic features & 0.465 & 0.511 & 0.450 & 0.422 \\
    \hline
europarl stacking, en only & 0.461 & 0.506 & 0.444 & 0.414 \\
    \hline
europarl stacking, en only +syntactic & 0.465 & 0.511 & 0.450 & 0.422 \\
    \hline
europarl stacking, 5 languages & 0.460 & 0.505 & 0.444 & 0.414 \\
    \hline
europarl stacking, 5 languages +syntactic & 0.464 & 0.510 & 0.449 & 0.422 \\
    \hline
    \hline

    \multicolumn{5}{|l|}{maxent l2} \\
    \hline
    baseline features & 0.475 & 0.524 & 0.458 & 0.431 \\
    \hline
    +all syntactic features & 0.480 & 0.530 & 0.465 & 0.441 \\
    \hline
europarl stacking, en only & 0.476 & 0.525 & 0.458 & 0.433 \\
    \hline
europarl stacking, en only +syntactic & 0.481 & 0.530 & 0.466 & 0.442 \\
    \hline
europarl stacking, 5 languages & 0.477 & 0.526 & 0.461 & 0.435 \\
    \hline
europarl stacking, 5 languages +syntactic & 0.482 & \emph{0.531} & 0.467 & \emph{0.444} \\
    \hline
    \hline

    \multicolumn{5}{|l|}{random forest} \\
    \hline
    baseline features & 0.481 & 0.520 & 0.464 & 0.424 \\
    \hline
    +all syntactic features & 0.486 & 0.527 & 0.471 & 0.434 \\
    \hline
europarl stacking, en only & 0.482 & 0.522 & 0.464 & 0.425 \\
    \hline
europarl stacking, en only +syntactic & 0.486 & 0.527 & 0.472 & 0.436 \\
    \hline
europarl stacking, 5 languages & 0.483 & 0.523 & 0.466 & 0.428 \\
    \hline
europarl stacking, 5 languages +syntactic & \emph{0.487} & 0.527 & \emph{0.473} & 0.438 \\
    \hline
  \end{tabulary}
  \end{centering}
  \caption{Results for stacking with Europarl.}
  \label{fig:europarl-stacking-results}
\end{figure*}

\subsection{Results: classifier stacking with Bibles}

In Figure \ref{fig:bible-stacking-results}, we see our results for the
experiments with stacking using Bibles as bitext.


\begin{figure*}
  \begin{centering}
  \begin{tabulary}{\textwidth}{|R|L|L|L|L|}
    \hline
    classifier & es-gn regular & es-gn non-null & es-qu regular & es-qu non-null \\

    \hline
    MFS    & 0.456 & 0.498 & 0.435 & 0.391 \\
    \hline
    \hline

    \multicolumn{5}{|l|}{maxent l1} \\
    \hline
    baseline features & 0.461 & 0.506 & 0.444 & 0.414 \\
    \hline
    +all syntactic features & 0.465 & 0.511 & 0.450 & 0.422 \\
    \hline
bible stacking, en only & 0.463 & 0.508 & 0.446 & 0.416 \\
    \hline
bible stacking, en only +syntactic & 0.466 & 0.513 & 0.451 & 0.425 \\
    \hline
bible stacking, 5 languages & 0.466 & 0.512 & 0.449 & 0.419 \\
    \hline
bible stacking, 5 languages +syntactic & 0.469 & 0.516 & 0.453 & 0.427 \\
    \hline
    \hline

    \multicolumn{5}{|l|}{maxent l2} \\
    \hline
    baseline features & 0.475 & 0.524 & 0.458 & 0.431 \\
    \hline
    +all syntactic features & 0.480 & 0.530 & 0.465 & 0.441 \\
    \hline
bible stacking, en only & 0.478 & 0.527 & 0.461 & 0.435 \\
    \hline
bible stacking, en only +syntactic & 0.482 & 0.532 & 0.467 & 0.444 \\
    \hline
bible stacking, 5 languages & 0.484 & 0.533 & 0.468 & 0.443 \\
    \hline
bible stacking, 5 languages +syntactic & 0.487 & \textbf{0.538} & 0.472 & \textbf{0.450} \\
    \hline

    \multicolumn{5}{|l|}{random forest} \\
    \hline
    baseline features & 0.481 & 0.520 & 0.464 & 0.424 \\
    \hline
    +all syntactic features & 0.486 & 0.527 & 0.471 & 0.434 \\
    \hline
bible stacking, en only & 0.484 & 0.525 & 0.466 & 0.428 \\
    \hline
bible stacking, en only +syntactic & 0.488 & 0.529 & 0.473 & 0.439 \\
    \hline
bible stacking, 5 languages & 0.488 & 0.530 & 0.470 & 0.434 \\
    \hline
bible stacking, 5 languages +syntactic & \textbf{0.492} & 0.533 & \textbf{0.476} & 0.441 \\
    \hline
  \end{tabulary}
  \end{centering}
  \caption{Results for stacking with Bibles.}
  \label{fig:bible-stacking-results}
\end{figure*}

\subsection{Results: classifier stacking with both Europarl and Bibles}

Finally, in Figure \ref{fig:both-stacking-results}, we show the results for
experiments where we used stacking features from both Bibles and the Europarl
corpus. The results in general were quite similar those we saw with using Bible
stacking features alone; the addition of the features from classifiers trained
on Europarl did not help.

\begin{figure*}
  \begin{centering}
  \begin{tabulary}{\textwidth}{|R|L|L|L|L|}
    \hline
    classifier & es-gn regular & es-gn non-null & es-qu regular & es-qu non-null \\

    \hline
    MFS    & 0.456 & 0.498 & 0.435 & 0.391 \\
    \hline
    \hline

    \multicolumn{5}{|l|}{maxent l1} \\
    \hline
    baseline features & 0.461 & 0.506 & 0.444 & 0.414 \\
    \hline
    +all syntactic features & 0.465 & 0.511 & 0.450 & 0.422 \\
    \hline
both stacking, 5 languages & 0.465 & 0.511 & 0.448 & 0.419 \\
    \hline
both stacking, 5 languages +syntactic & 0.468 & 0.516 & 0.452 & 0.426 \\
    \hline
    \hline

    \multicolumn{5}{|l|}{maxent l2} \\
    \hline
    baseline features & 0.475 & 0.524 & 0.458 & 0.431 \\
    \hline
    +all syntactic features & 0.480 & 0.530 & 0.465 & 0.441 \\
    \hline
both stacking, 5 languages & 0.484 & 0.533 & 0.469 & 0.444 \\
    \hline
both stacking, 5 languages +syntactic & 0.487 & \textbf{0.538} & 0.473 & \textbf{0.450} \\
    \hline
    \hline

    \multicolumn{5}{|l|}{random forest} \\
    \hline
    baseline features & 0.481 & 0.520 & 0.464 & 0.424 \\
    \hline
    +all syntactic features & 0.486 & 0.527 & 0.471 & 0.434 \\
    \hline
both stacking, 5 languages & 0.489 & 0.530 & 0.471 & 0.434 \\
    \hline
both stacking, 5 languages +syntactic & \textbf{0.492} & 0.533 & \emph{0.475} & 0.443 \\
    \hline
    \hline

  \end{tabulary}
  \end{centering}
  \caption{Results for stacking with Bibles.}
  \label{fig:both-stacking-results}
\end{figure*}

\section{Discussion}

