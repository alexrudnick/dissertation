\chapter{Learning from Multilingual Data}
\label{chap:multilingual}

\section{Using multilingual evidence}
For some languages, such as English, Spanish, and the other languages of the
European Union, we have multiple bitext corpora available, covering several
different language pairs with the same source language. Through the Europarl
corpus \cite{europarl}, for example, we have bitext corpora in which English is
paired with Spanish, German, French, and 17 other European languages.
We would like to be able to make use of evidence from all of these corpora when
translating into any particular target language, ideally even when the corpora
involved are not all mutually parallel.
Each corpus may contain useful examples of a given source language word,
and senses of that word may be lexicalized in varying, non-overlapping ways in
the different target languages.
This approach is particularly informed by the ongoing work of Els Lefever
\emph{et. al} (see especially \cite{lefever-hoste-decock:2011:ACL-HLT2011}),
although their technique requires an entire machine translation system to
perform CL-WSD, which seems unwieldy when we want to use CL-WSD as a
subcomponent of a machine translation system.

We want a CL-WSD system to be able to learn the relationships between the
senses of a given source word that are lexicalized in different ways in various
languages.
Two target languages may happen to surface the same sense distinctions, perhaps
due to being related languages, or simply by coincidence.
Alternatively, a combination of translations into several languages may provide
evidence for a certain lexical choice in the target language of interest.

In early 2013, we developed a prototype CL-WSD system that makes use of
multilingual evidence \cite{rudnick-liu-gasser:2013:SemEval-2013} and produced
some of the best results in a SemEval shared task on CL-WSD \cite{task10}.
Concretely, teams undertaking the task had to translate polysemous nouns from
English into five other European languages.
In our SemEval paper, we presented three variations on the approach:
a straightforward classification approach (without multilingual evidence), a
classifier stacking approach, and a graphical model system based on loopy
belief propagation over Markov networks.

The simplest system presented in that work, based on a maximum entropy
classifier, simply extracted features from a window around the English noun to
be translated and used these to make a prediction. The same system could also
return a probability distribution over target language words or phrases. This
simple system was used as a subcomponent in the two more sophisticated systems.

\subsection{Classifier stacking}
The classifier stacking approach, in order to predict the translation of a word
into a given target language, uses translations into the other four available
target languages as features.
For example, to translate an English word into Spanish, we would give the
classifier features encoding that word's translations into French, Italian,
Dutch and German.
At training time, since we were provided with parallel translations for all six
languages, we found the correct answers in the parallel corpora, but for test
sentences, the translations were not available, so they were predicted with the
simpler non-multilingual classifiers.

The classifier stacking approach could likely be improved by adding features
derived from more sources.
We would like to try adding classifiers trained on the other Europarl
languages, as well as completely different corpora; this would require that for
the stacked classifier, we would train on predicted translations rather than
This approach only requires that the first-layer classifiers make \emph{some}
prediction based on text in the source language;
they need not be trained from the same source text, depend on the same
features, or even output words as labels. Monolingual WSD systems seem likely
to help out, as do Brown-style word clusters \cite{brown1992class}.

\subsection{Markov Networks}
In our SemEval entry, we also investigated a Markov network (or ``Markov Random
Field") approach, building a network of interacting variables (see Figure
\ref{fig:pentagram}) to solve the classification problem for all five target
languages jointly.

The network has nodes that correspond to the distributions produced by the
simple classifiers, given an input sentence, and edges with pairwise potentials
that are derived from the joint probabilities of target language labels
occurring together in the training data. 
Thus the task of finding the optimal translations into five languages jointly
is framed as a MAP inference problem, where we try to maximize the joint
probability of all five variables, given the evidence of the features extracted
from the source language sentence.
The inference process is performed using loopy belief propagation
\cite{DBLP:conf/uai/MurphyWJ99}, which is an approximate but tractable
inference algorithm that, while giving no guarantees, often produces good
solutions in practice.
We used the formulation for pairwise Markov networks that passes messages
directly between the nodes rather than first constructing a cluster graph,
which is described in \cite[\S 11.3.5.1]{Koller+Friedman:09}.

\begin{figure}
  \begin{center}
  \includegraphics[width=5cm]{pentagram.pdf}
  \end{center}
  \caption{The network structure used in the MRF system: a complete graph with
  five nodes, in which each node represents the random variable for the
  translation into a target language.}
  \label{fig:pentagram}
\end{figure}

Our initial approach for setting the weights in the Markov network requires,
for each pair of languages, a parallel corpus for that language. It thus seems
less easily extensible than the classifier stacking approach, in which it is
clear how we can include all kinds of heterogeneous information.
The benefit of the Markov network approach is that it takes seriously the
uncertainty present in the predictions of each of the component classifiers,
solving the entire problem jointly.
The Markov network approach may also play a part in our future CL-WSD systems.

