\chapter{Evaluation}
\label{chap:evaluation}

In order to evaluate our CL-WSD approaches and their effects on translation
systems, we will need to have some metrics for comparison between our
techniques, sensible baselines, and results reported by other researchers.

One clear \emph{in vitro} approach for measuring the success of a CL-WSD system
will be the metrics provided by the SemEval shared tasks
\cite{lefever-hoste:2009:SEW,task10}; measuring our performance on these test
sets will provide a clear comparison between our work, several other CL-WSD
systems from the past few years, and the results posted by Els Lefever's
ParaSense system.

These test sets are limited in scope, however, and will only demonstrate our
performance for translating individual polysemous English nouns, ignoring the
problem of translating the rest of the sentence.
Furthermore, our broader practical goal is to use Spanish as a source language,
and the resources that we can use to analyze an input English sentence will be
different from those for Spanish.

An approach that we took recently \cite{rudnick-gasser:2013:HyTra-2013} was to
simply report classification accuracy for the CL-WSD task of labeling input
Spanish sentences with Guarani stems, from a held-out test set sampled from the
Bible.
This will be closer to an accurate assessment of a lexical selection system
trying to translate every word in running text with a realistically large
vocabulary. As we collect more bitext, we will be able to construct test sets
from a broader domain.

We will additionally want to conduct \emph{in vivo} evaluations of our CL-WSD
techniques while they are in use for lexical selection in running MT systems.
Thus we will use standard approaches for evaluating MT, such as BLEU and METEOR
scores, and human evaluations, likely conducted online so that fluent Guarani
speakers will be able to participate.
