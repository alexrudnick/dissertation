\chapter{Overview}

\begin{centering}
\emph{A thing with just one meaning has scarcely any meaning at
all.} \\
-- Marvin Minsky, \emph{Society of Mind} \cite{minsky1988society} \\
\end{centering}
\bigskip

In this dissertation, I investigate techniques for lexical selection in
machine translation, with the goal of supporting language pairs for which 
there is little training data available for the target language. I propose
that cross-lingual word sense disambiguation (CL-WSD) is a feasible and
practical means for lexical selection in this setting. I demonstrate
this with translation experiments covering several language pairs, focusing on
translating from Spanish to Guarani (a co-official language of Paraguay) and
Spanish to Quechua (broadly spoken around the Andes mountain range), and
including some experiments involving English.

I describe new CL-WSD approaches, including the use of unsupervised clustering
on source-language text and multilingual corpora for the source language as a
source of classification features. I also evaluate their effectiveness in
practice. All of these techniques have been implemented in a new CL-WSD
software system called Chipa
\footnote{Chipa the software is named for chipa the snack food, popular in many
  parts of South America. It is a cheesy bread made from cassava flour, often
  served in a bagel-like shape in Paraguay.  Also \emph{chipa} means 'rivet,
  bolt, screw' in Quechua; something for holding things together.  The software
is available at \\ \url{http://github.com/alexrudnick/chipa} under the GPL.}.
We also integrate Chipa into two machine translation systems of different
architectures to show its practical applicability.

The research presented here is motivated in large part to allow primarily
rule-based machine translation (RBMT) systems to make use of any available
bitext training data, so that they can improve their translations without their
human authors having to write new rules.  Despite the enormous
success of statistical machine translation (SMT) approaches over the past
twenty years, they typically require large bitext corpora to be useful. RBMT
approaches remain attractive for many language pairs due to a simple dearth of
training data, and there is an active RBMT community working on developing
machine translation systems for the world's under-resourced and
under-represented languages.

My hope for this work is that it will provide techniques and tools useful for
anyone working on machine translation into these languages, whatever the
architecture of their MT software, allowing them to make use of bitext as it
becomes available for their language pair. This combination of strategies is
what is meant by ``hybrid" in this work. We would like to combine existing
RBMT systems with statistical methods where possible, neither throwing away the
efforts and insights of linguists, nor ignoring the benefits of machine
learning and corpus-based approaches.

\section{WSD in Machine Translation}
%% TODO Get italics vs quoting straight here. When do we use which?
To appreciate the word-sense disambiguation problem embedded in machine
translation, consider for a moment the different senses of ``have" in
English. In \emph{have a sandwich}, \emph{have a bath}\footnote{Compare with
``take a bath", which has an additional metaphorical interpretation meaning
``serious financial loss"}, \emph{have an
argument}, and even \emph{have a good argument}, the meaning of the verb ``to
have" is quite different. It would be surprising for our target language,
especially if it is not closely related, to use a single light verb in
all of these distinct contexts. Even in a language as closely related as
Spanish, we see these English expressions rendered with some different
verbs: \emph{tomar un bocadillo} but \emph{tener un argumento}, for example.
By cross-lingual word sense disambiguation (CL-WSD), I mean exactly this
problem: we consider \emph{tomar} and \emph{tener} to be two different
senses of \emph{have} -- there are certainly more, for a verb like \emph{have}
-- and the CL-WSD software must label each source-language word with its
correct translation.

Another concrete example, due to Annette Rios \cite{rudnick:saltmil2014},
surfaces when translating from Spanish to Quechua. 
Here we see different lexicalization patterns in Spanish and Quechua
for transitive motion verbs. The Spanish lemmas contain information about the
path of the movement, e.g. {\em traer} - 'bring (here)' vs. {\em llevar} -
'take (there)'. Quechua roots, on the other hand, use a suffix ({\em -mu}) to
express direction, but instead lexicalize information about the manner of
movement and the object that is being moved. Some of these verbs are as
follows:

\begin{itemize}
\renewcommand{\labelitemii}{$\bullet$}
 \item[] general motion verbs:
 \begin{itemize}
 \item {\em pusa-(mu-)}: `take/bring a person'
 \item {\em apa-(mu-)-}: `take/bring an animal or an inanimate object'
 \end{itemize}
 \item[] motion verbs with manner:
 \begin{itemize}
 \item {\em marq'a-(mu-)}: `take/bring something in one's arms'
 \item {\em q'ipi-(mu-)}:  `take/bring something on one's back or in a bundle'
 \item {\em millqa-(mu-)}: `take/bring something in one's skirts'
 \item {\em hapt'a-(mu-)}: `take/bring something in one's fists'
 \item {\em lluk'i-(mu-)}: `take/bring something below their arms'
 \item {\em rikra-(mu-)}:  `take/bring something on one's shoulders'
 \item {\em rampa-(mu-)}:  `take/bring a person holding their hand'
 \end{itemize}
\end{itemize}

%% TODO Get italics vs quoting straight here. When do we use which?
The correct translation of Spanish {\em traer} or {\em llevar} into Quechua
thus depends on the object being moved, and more descriptive translations can
be chosen, given context clues.
Furthermore, different languages simply make
different distinctions about the world. The Spanish \emph{hermano} 'brother',
\emph{hijo} 'son' and \emph{hija} 'daughter' all translate to different Quechua
terms based on the person with respect to whom we are describing the referent;
a daughter relative to her father is \emph{ususi}, but when described relative
to her mother, \emph{warmi wawa} \cite{academiamayor}.

Chipa, then, must somehow make these distinctions automatically if it is to
help translation systems generate valid Quechua. It is possible to do this with
hand-crafted lexical selection rules, or by treating this as a machine learning
problem and basing lexical selections on the available data, or by some
combination of the two. 

Lexical ambiguity presents a daunting challenge for rule-based machine
translation systems. Several translations of a given word may all be
syntactically valid in context.
Even when choosing among near-synonyms, we would like to respect the
selectional preferences of the target language so as to produce
natural-sounding output text.

Writing lexical selection rules by hand, while possible in principle and an
approach taken by many MT system implementors, is tedious and error-prone.
Bilingual informants, if available, may not be able to enumerate the contexts
in which they would choose one alternative over another, and there will be many
thousands of words that might need to be translated in any given source
language. Thus we would like to learn from corpora when possible.

Given a sufficient bitext corpus, we can discover the different possible
translations for each source-language word, and by applying supervised
learning, we can learn how to discriminate between them.

We do not have very large sentence-aligned bitext corpora for most language
pairs, however, and will have to make do with the Bible (though other bitext is
absolutely welcome and valuable!), which is an acceptable
starting point containing many commonly used words.\footnote{The
Bible has been translated into a substantial fraction of the world's languages
and contains more than thirty thousand verses (roughly, sentences) in
a variety of text genres. For an overview of using the Bible for multilingual
NLP applications, please see Resnik et al.
\cite{DBLP:journals/lre/ResnikOD99} and Mayer and Cysouw
\cite{MAYER14.220.L14-1215}.} In the future, we would like to see larger
bitext corpora constructed for under-resourced language pairs by their
respective communities. Our research group has been working on this problem in
parallel, though that is not the focus of this dissertation.

The major contributions of this work will be new approaches for CL-WSD,
demonstrations that they can be used in practice for translating from
higher-resourced languages into lower-resourced ones, and demonstrations that
they can be integrated into practical machine translation systems, thus
improving word choice.
Additionally, on a practical level we will develop a suite of reusable 
open-source software, including the CL-WSD engine and demo MT systems of
different architectures for for Spanish-Guarani and Spanish-Quechua.

All of the software used in this dissertation has been developed in public
repositories, and is freely available online at
\url{http://github.com/hltdi} and \url{http://github.com/alexrudnick}.

\section{Thesis Statement}
Cross-lingual word sense disambiguation is a feasible and practical
means for lexical selection in a hybrid machine translation system for a
language pair with relatively modest resources.

\section{Questions to Address}
\begin{enumerate}
\item Can we use monolingual resources from the source language to improve
accuracy on the cross-lingual word sense disambiguation task?
\item Can we use multilingual resources, such as bitext corpora pairing the
source language with languages other than the current target language?
\item Which kinds of machine translation systems can benefit from CL-WSD?
\end{enumerate}

\section{Dissertation Structure}
In the following chapters, I will discuss the relevant background information
and investigate these questions, as follows.

\begin{itemize}
\item Chapter \ref{chap:background} gives some background on word sense
disambiguation, its history, and its applications in machine translation, as
well as on Spanish and Guarani, the languages of Paraguay, and some of the
social context for this translation pair.
\item Chapter \ref{chap:relatedwork} reviews some recent work on CL-WSD and its
applications.
\item Chapter \ref{chap:evaluation} presents the data sets used in this work,
and our evaluation metrics for this project.
\item Chapter \ref{chap:baseline} discusses the basic architecture of the
Chipa system and its performance on the previously presented evaluation metrics.
\item Chapter \ref{chap:monolingual} explores some approaches for learning from
the available resources for Spanish to give us better representations for
translating out of it.
\item Chapter \ref{chap:multilingual} expands our approaches to include the
multilingual resources for Spanish, integrating CL-WSD classifiers for
languages other than Guarani and Quechua.
\item Chapter \ref{chap:integration} presents practical applications of Chipa;
here we will see how to integrate CL-WSD software into different running
machine translation systems.
\item Finally, in Chapter \ref{chap:conclusions}, we will conclude and describe
possible work for the future.
\end{itemize}

\section{Previously Published Work}
This dissertation draws heavily from several previously published papers, in
which my coauthors and I explore early versions of some of the ideas and
techniques described here. They are as follows:

\begin{itemize}
\item Rudnick 2011, ``Towards Cross-Language Word Sense Disambiguation for
Quechua" \cite{rudnick:2011:RANLPStud}, in which I explored a simple CL-WSD
system for Quechua adjectives.
\item Rudnick, Liu and Gasser 2013, ``HLTDI: CL-WSD Using Markov Random Fields
for SemEval-2013 Task 10" \cite{rudnick-liu-gasser:2013:SemEval-2013}, a
SemEval submission in which we explored the use of multilingual evidence for
CL-WSD.
\item Rudnick and Gasser 2013, ``Lexical Selection for Hybrid MT with Sequence
Labeling" \cite{rudnick-gasser:2013:HyTra}, in which we explored using
Maximum-Entropy Markov Models (MEMMS) for CL-WSD for Spanish-Guarani.
\item Rudnick, Rios and Gasser 2014, ``Enhancing a Rule-Based MT System with
Cross-Lingual WSD" \cite{rudnick:saltmil2014}, in which we integrated Chipa
into an RBMT system for Spanish-Quechua and explored monolingual clustering to
learn from Spanish data.
\item Rudnick, Skidmore, Samaniego and Gasser 2014, ``Guampa: a Toolkit for
Collaborative Translation" \cite{RUDNICK14.151}, which describes our system for
building larger corpora through enlisting the Guarani-language activist
community to help. Guampa, or at least the idea of engaging the
language-activist and language-learning communities, figures largely into the
future of this work, rather than having been helpful for data gathering as of
yet.
\end{itemize}
