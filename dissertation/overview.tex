\chapter{Overview}
In this dissertation, I will investigate techniques for building a hybrid
machine translation (HMT) system for a language pair with relatively modest
resources.
I propose that cross-lingual word sense disambiguation (CL-WSD) is a feasible
and practical means for lexical selection in this setting, and will work to
demonstrate this by using it to construct a HMT system for Spanish and Guarani,
the co-official languages of Paraguay.
Along the way, I will develop new CL-WSD approaches, including the use of
multilingual evidence and sequence-labeling techniques, and integrate our
CL-WSD software into several different machine translation systems.

Lexical ambiguity presents a serious challenge for rule-based machine
translation (RBMT) systems, since many words will have multiple possible
translations in the target language. Moreover, several translations of a given
word may all be syntactically valid in context, but have significantly
different meanings. Even when choosing among near-synonyms, we would like to
respect the selectional preferences of the target language so as to produce
natural-sounding output text.

Writing lexical selection rules by hand is tedious and error-prone; bilingual
informants, if available, may not be able to enumerate the contexts in which
they would choose one alternative over another. Thus we would like to learn
from corpora when possible. However, for most language pairs, suitably large
sentence-aligned bitext corpora are not available, so creating and deploying a
translation system based on machine learning techniques will require collecting
a larger corpus. In order to support this work, we have ongoing collaborative
projects for crowdsourcing the collection of larger bitexts.

The major contributions of this work will be new approaches for CL-WSD,
integrating CL-WSD into MT systems, and building MT systems that target
morphologically rich language with few language resources.  Additionally, on a
practical level we will develop a suite of reusable open-source software
including a hybrid MT system for Spanish-Guarani, websites for crowdsourcing
bilingual data collection to a relatively small though engaged population, and
a freely available corpus of Spanish-Guarani bitext
\footnote{All development is being carried out in public repositories, and will
be linked from \url{http://hltdi.github.io} and
\url{http://github.com/alexrudnick}}.

%% In this work I will focus on Spanish and Guarani, the co-official languages of
%% Paraguay. I will describe novel approaches to CL-WSD, including using evidence
%% from multilingual sources and sequence labeling techniques. I will also
%% investigate the collection of a larger bilingual corpus through crowdsourcing
%% and the integration of the disambiguation techniques into a rule-based
%% translation engine, producing a hybrid rule-based/statistical translation
%% system.

\section{Thesis statement}
Cross-lingual word sense disambiguation (CL-WSD) is a feasible and practical
means for lexical selection in a hybrid machine translation system for a
language pair with relatively modest resources.

\section{Questions to address}
\begin{enumerate}
\item What are effective ways to build a practical machine translation system
targeting an under-resourced, morphologically rich language?
\item Which CL-WSD techniques are useful for this purpose?
\item Which kinds of MT systems can benefit from CL-WSD?
%% \item How can we collect bigger corpora to train our system?
\end{enumerate}
