This thesis argues that cross-lingual word sense disambiguation (CL-WSD) can be
used to improve lexical selection for machine translation when translating from
a resource-rich language into an under-resourced one, especially when relatively
little bitext is available.  In CL-WSD, we perform word sense disambiguation,
considering the senses of a word to be its possible translations into some
target language, rather than using a sense inventory developed manually by
lexicographers.

Using explicitly trained classifiers that make use of source-language context
and of resources for the source language can help machine translation systems
make better decisions when selecting target-language words. This is especially
the case when the alternative is hand-written lexical selection rules developed
by researchers with linguistic knowledge of the source and target languages, but
also true when lexical selection would be performed by a statistical machine
translation system, when there is a relatively small amount of available
target-language text for training language models.

In this work, I present the Chipa system for CL-WSD and apply it to the task of
translating from Spanish to Guarani and Quechua, two indigenous languages of
South America. We demonstrate several extensions to the basic Chipa system,
including techniques that allow us to benefit from the wealth of available
unannotated Spanish text and existing text analysis tools for Spanish, as well
as approaches for learning from bitext resources that pair Spanish with
languages unrelated to our intended target languages. Finally, I provide
proof-of-concept integrations of Chipa with existing machine translation
systems, of two completely different architectures.
