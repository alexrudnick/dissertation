\chapter{Background}
\label{chap:background}

XXX We need to decide, y'know, what goes into the background section.

What did we say in the outline?

\begin{itemize}
\item history of CL-WSD
\item related work
\item Guarani language
\item maybe Quechua and Amharic languages too!
\end{itemize}


\section{Hybrid MT}

In recent years, we have seen renewed interest in machine translation systems
that take into account syntactic structure, linguistic knowledge, and semantic
representations.
Hopefully, these will provide better translation for language pairs with
significant reordering or syntactic divergences, and where one or both of the
languages has rich morphology.
The boundaries between rule-based and statistical MT systems are becoming
increasingly blurred, and hybrid systems are being developed in both
directions, with RBMT systems incorporating components based on machine
learning, as well as SMT systems making use of linguistic knowledge for
morphology and syntax.

Additionally, for most of the world's language pairs, there is simply no large
bitext corpus available, so training a purely statistical machine translation
system is infeasible.
Thus, while SMT approaches have had great success, and drastically changed the
machine translation landscape since the 1990s, RBMT approaches are still
relevant for many language pairs.

We would like for RBMT or hybrid systems, once developed, to be able to make
use of any bitext on hand.  Like SMT systems, they should be able to produce
better translations as larger corpora become available, without additional code
changes.
In this dissertation work, we will develop a software package called
chipa, which will do just that: %% XXX: better wording?
it will integrate with a variety of machine translation systems to help them
use any available bitext to make better word choices.







\section{History of CL-WSD}


ParaSense, the CL-WSD system developed by Els Lefever
\cite{lefever-hoste-decock:2011:ACL-HLT2011}, takes into account evidence from
several different parallel corpora.
For predicting the translation of a source word into
any particular target language, ParaSense creates
bag-of-words features from the translations of the input sentence into every
other language that it knows about. As of this writing, ParaSense handles
translation from English into French, Spanish, Italian, Dutch and German.
Given corpora that are parallel over many languages, such as Europarl, this is
straightforward at
training time. However, at testing time it requires a complete MT system for
each of the four other languages, which seems computationally prohibitive. In
the past, ParaSense has simply called out to the Google Translate API to
generate the bag-of-words features required for test sentences. This seems
unwieldy, and thus in our work, we learn from several parallel corpora but
require neither a locally running MT system nor access to an online translation
API.

To our knowledge, there has not been other work on framing CL-WSD for all words
in an input sentence as a sequence labeling problem. However, in monolingual
WSD, Molina \textit{et al.} \cite{DBLP:conf/iberamia/MolinaPS02} have made
use of HMMs for WSD. 


\section{CL-WSD for Lexical Selection}
While most SMT systems do not make use of an explicit WSD module, recently
there has been work on adding WSD classifiers in to statistical MT systems.
Particularly, Carpuat and Wu have shown how to use CL-WSD, or more broadly,
cross-lingual phrase sense disambiguation, to improve modern phrase-based SMT
systems
\cite{carpuatpsd,carpuat-wu:2007:EMNLP-CoNLL2007,carpuat2008evaluation}. In
Carpuat's work, classifiers are used to label multi-word expressions (phrases,
in the phrase-based SMT sense) with target language phrases. She demonstrates
how this is more appropriate in an an SMT setting than simply labeling
individual words with WordNet synsets, as had previously been attempted, and
showed significant improvements on a Chinese-English translation task.

There has also been work on using discriminative MaxEnt models to adapt
the ``forward" translation model of an SMT system, using richer
source-language context features \cite{vzabokrtsky-popel-marevcek:2010:WMT}.
Somewhat interestingly, while it has the same effect, this work does not
describe itself in terms of word sense disambiguation.

%% text from the HyTra paper
Although they did not present a complete MT system, there has also been work
on using WSD techniques for translation into lower-resourced languages, such as
the English-Slovene language pair, as in
\cite{vintar-fivser-vrvsvcaj:2012:ESIRMT-HyTra2012}. 

%%The Apertium team has a particular practical interest in improving lexical
%%selection in RBMT; they recently have been developing
%%a new system, described in \cite{tyers-fst}, that learns finite-state
%%transducers for lexical selection from the available parallel corpora. It is
%%intended to be both very fast, for use in practical translation systems, and
%%to produce lexical selection rules that are understandable and modifiable by
%%humans.

Francis Tyers, in his dissertation work \cite{tyers-dissertation},
provides an overview of lexical selection systems and describes methods for
learning lexical selection rules based on available parallel corpora.
These rules make reference to the lexical items and parts of speech surrounding
the word to be translated. Once learned, these rules are intended to be
understandable and modifiable by human language experts. For practical use in
the Apertium machine translation system, they are compiled to finite-state
transducers.


\section{Online Language Tools for Guarani}
There are currently very few online language tools for Guarani; some of the
best available ones are iGuarani \footnote{\url{http://iguarani.com/}}, a
searchable online dictionary and gisting translation system developed by young
Paraguayan programmer Diego Alejandro Gavilán.

There is also a small searchable online dictionary developed by Wolf Lustig
\footnote{\url{http://www.uni-mainz.de/cgi-bin/guarani2/dictionary.pl}},
which has versions in Spanish, German, and English. He has graciously made this
dictionary available for our use, and this will provide a starting point for
our Spanish-Guarani translation system.

The Guarani activist and education community has a presence on the Web,
although a startlingly large fraction of this presence is a single author,
Dr. David Galeano Olivera, the president of the \emph{Ateneo de Lengua y
Cultura Guaraní}. He has collected a number of resources both about and in the
Guarani language
\footnote{\url{http://cafehistoria.ning.com/profiles/blogs/la-lengua-guarani-o-avanee-en}},
some in Spanish and some in Guarani, and will likely continue to do so.





Here classifiers are trained with the scikit-learn machine learning package
\cite{scikit-learn}, using logistic regression (also known as ``maximum
entropy") with the default settings and the regularization constant set to
$C=0.1$. We also use various utility functions from NLTK \cite{nltkbook}. 

For this work, we use familiar features for text classification: the
surrounding lemmas for the current token (three on either side) and the
bag-of-words features for the entire current sentence. We additionally include,
optionally, the Brown cluster labels (see below for an explanation),
both for the immediate surrounding context and the entire sentence.
We suspect that more feature engineering, particularly making use of syntactic
information and surface word forms, will be helpful in the future.

\begin{figure}[t!]
  \begin{itemize}
    \item lemmas from surrounding context (three tokens on either side)
    \item bag of lemmas from the entire sentence
    \item Brown cluster labels from surrounding context
    \item bag of Brown cluster labels from the entire sentence
  \end{itemize}
\caption{Features used in classification}
\label{fig:features}
\end{figure}

\section{Related Work}

\subsection{early SMT with WSD}
%% Brown et al
Framing the resolution of lexical ambiguities in machine translation
as an explicit classification
task has a long history, dating back at least to early SMT work at IBM
\cite{Brown91word-sensedisambiguation}.

An early paper by IBM researchers \cite{Brown91word-sensedisambiguation},
outlines the CLWSD task in a strikingly similar way, as a WSD task where the
possible senses of a word are extracted from statistical alignments learned
over a bitext corpus. Brown \textit{et al.} report significant translation
quality improvements through the use of their WSD system, over a small
hand-evaluated set of test sentences.


\subsection{PB-SMT with phrase sense disambiguation}
%% Carpuat and Wu
More recently, Carpuat and Wu have
shown how to use classifiers to improve modern phrase-based SMT systems
\cite{carpuatpsd}.

Recent years have seen a resurgence of interest in the integration of
word-sense disambiguation techniques into machine translation.  We suspect that
WSD will be especially useful for translating into under-resourced and
morphologically rich languages, for which good language models are likely to be
sparse. Before the work of 
Carpuat and Wu~\cite{improvingsmtwsd}, it was apparently unclear whether
WSD was necessary or helpful for a state-of-the-art statistical MT system;
lexical choice among possible translations for a given word can often be
handled by the language model for the target language, simply due to
collocations of appropriate words in the training data.


\subsection{CL-WSD at SemEval}
CL-WSD has received enough attention to warrant shared tasks at recent SemEval
workshops; the most recent running of the task is described by Lefever and
Hoste \cite{task10}.
In this task, participants are asked to translate twenty different polysemous
English nouns into five different European languages, in a variety of contexts.


\subsection{multilingual evidence for CL-WSD}
%% ParaSense
Lefever \emph{et al.}, in work on the ParaSense system
\cite{lefever-hoste-decock:2011:ACL-HLT2011}, produced top results for
this task with classifiers trained on local contextual features, with the 
addition of a bag-of-words model of the translation of the complete source
sentence into other (neither the source nor the target) languages. At training
time, the foreign bag-of-words features for a sentence are extracted from
available parallel corpora, but at testing time, they must be
estimated with a third-party MT system, as they are not known a priori.
This work has not yet, to our knowledge, been integrated into an MT system
on its own.

Lefever \textit{et al.} recently described a novel approach to WSD, making use
of evidence from several languages at once to disambiguate English-language
source sentences. This is done by building artificial parallel corpora in
several languages, on demand, with the Google Translate API. They outperform
the previous state-of-the art systems on the SemEval 2010 shared task 13
\cite{lefever-hoste-decock:2011:ACL-HLT2011}.

In our earlier work, we prototyped a system that addresses some of the issues
with ParaSense, requiring more modest software infrastructure for feature
extraction while still allowing CL-WSD systems to make use of several mutually
parallel bitexts that share a source language
\cite{rudnick-liu-gasser:2013:SemEval-2013}.

\subsection{WSD for lower-resourced languages}
However, there has been work recently on using WSD techniques for translation
into lower-resourced languages, such as the English-Slovene language pair, as
in \cite{vintar-fivser-vrvsvcaj:2012:ESIRMT-HyTra2012}. 

Dinu and Kübler~\cite{Dinu07} have addressed the problem of monolingual WSD for
a lower-resourced language, particularly Romanian. In their work, they describe
an instance-based approach in which a relatively small number of features is
used quite effectively. In other work on lower-resourced languages,
Sarrafzdadeh \textit{et al.} have investigated a version of the Lesk algorithm
for Farsi.

%% XXX
One sort of interesting point here is that we are not doing WSD on a
low-resourced language. This is ultimately WSD on Spanish with a sense
inventory that we automatically discover from cross-lingual evidence.

\subsection{lexical selection in low-resource settings}
Francis Tyers, in his dissertation work \cite{tyers-dissertation},
provides an overview of lexical selection systems and describes methods for
learning lexical selection rules based on available parallel corpora. These
rules make reference to the lexical items and parts of speech surrounding the
word to be translated. Once learned, these rules are intended to be
understandable and modifiable by human language experts. For practical use in
the Apertium machine translation system, they are compiled to finite-state
transducers.

The Apertium team has a particular practical interest in improving lexical
selection in RBMT; they recently have been developing
a new system, described in \cite{tyers-fst}, that learns finite-state
transducers for lexical selection from the available parallel corpora. It is
intended to be both very fast, for use in practical translation systems, and
to produce lexical selection rules that are understandable and modifiable by
humans.

%% SQUOIA extensions: relevant to Hybrid MT
Rios and G\"{o}hring \cite{riosgonzales-gohring:2013:HyTra} describe
earlier work on extending the SQUOIA MT system with machine learning modules.
They used classifiers to predict the target forms of verbs in cases where the
system's hand-crafted rules cannot make a decision based on the current
context.

\subsection{Sequence models for word-sense disambiguation}
To our knowledge, there has not been work specifically on sequence labeling
applied to lexical selection for RBMT systems.

Outside of the CL-WSD setting, there has been work on framing all-words WSD as
a sequence labeling problem. Particularly, Molina \textit{et al.}
\cite{DBLP:conf/iberamia/MolinaPS02} have made use of HMMs for all-words
WSD in a monolingual setting.

We have also done some previous work on CL-WSD for translating into indigenous
American languages; an earlier version of Chipa, for Spanish-Guarani, made use
of sequence models to jointly predict all of the translations for a sentence at
once \cite{rudnick-gasser:2013:HyTra}.



\subsection{Translation into Morphologically Rich Languages}
Chris Dyer's recent paper at EMNLP
\cite{chahuneau:2013:emnlp}

Talk about prediction for morphology.
\cite{toutanova-suzuki-ruopp:2008:ACLMain}

Also factored models...
\cite{yeniterzi-oflazer:2010:ACL}
