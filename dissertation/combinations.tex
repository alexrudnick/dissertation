\chapter{Combining Approaches}
\label{chap:combinations}

The deal with morphogen \cite{chahuneau:2013:emnlp}
is that they, on demand, populate the phrase table with brand new phrases with
correct morphology already there. Then the decoder selects among the PT entries
with the regular infrastructure that you would have used anyway for SMT.

So in Tereré, we don't want to use TL lemmas as our label set; we want to use
the PT entries.

\section{System Combination and Other Future Directions}
There are a number of future directions that we would like to investigate, in
addition to the ones mentioned previously.
Notably, we will do significant feature engineering; so far, the features we
have used for our classifiers have been limited to the tokens in a small window
around the current source word and their part-of-speech tags.

For the Spanish-Guarani language pair, we will be able to make use of
the comparatively large set of resources for analyzing the input Spanish.
We could extract many different features, perhaps syntactic ones through a
source language parser, word-cluster features for each input word, WordNet
synset labels (when available), bag-of-words features for the entire sentence,
or even discourse or document-level features where appropriate.

%% not actually a great idea??
%% We will also develop ways to integrate different CL-WSD algorithms with each
%% other and with their containing MT systems; one obvious approach would be to do
%% classifier stacking from within an MEMM.

%% XXX maybe move this to future work? how could you integrate this with NMT,
%% though? Does that even make sense?
%% We could also imagine having an MT system invoking several different CL-WSD
%% systems and using them in an ensemble-like way.
%% Perhaps the CL-WSD systems could vote, with each of their recommendations
%% weighted using MERT, and leaving the final choice to the decoder.

%% This can probably just be moved into a future work section.
We will also need to address both multi-word expressions and morphology and how
they interact with the CL-WSD system.
Thus far, we have assumed one-to-many alignments, and labeled each source word
with zero or more lemmatized target words or word stems.
However in practice, we will want to make use of multi-word expressions, and
our translation systems will need to generate appropriately inflected target
language text.

%The question we were trying to address is like:
%what if your n-gram model over labels for your source language has one idea
%(this is the next likely label), and your emission probabilities say something
%else (this is the label most likely to generate the observed input word)...
%well, we could imagine just making several different decisions and letting the
%decoder sort it out. We're doing that already actually, by including
%phrase-table probabilities and a language model.
%Are we just converging on the idea of a log-linear model again?
%But the point here is that we don't have to have just one CL-WSD system. We can
%have a bunch and let them fight it out. Hopefully this will help!
