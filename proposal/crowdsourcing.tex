\section{Acquiring larger bitext corpora}
To gather a larger training corpus, both of Spanish-Guarani bitext and
monolingual Guarani text, we plan to build two websites. One website will be a
collaborative online space for building translations of documents, primarily
from Spanish to Guarani, although the other direction will be useful as well.
The second will be a repository of Guarani and bilingual documents,
complete with a search engine and the ability to upload new documents.
Initial designs for both of these sites were done as a master's project in HCI
by Alberto Samaniego\footnote{\url{http://albsama.com}}, who will hopefully
continue collaborating on this project from his native Paraguay. We also hope
to enlist other collaborators, definitely from Paraguay and perhaps from
Indiana and the broader open-source world as well.

\subsection{Collecting Guarani Documents}
The first website we will develop is called ``Tahekami", which means
\emph{let's search together} in Guarani.
On Tahekami, users will be able to search for documents
using a search feature that takes into account the morphological richness of
Guarani, browse by tag, and upload new documents.

An initial version of this site is already well underway \footnote{It is being
developed at \url{http://github.com/hltdi/gn-documents}}. We have a working
search engine based on the Whoosh library
\footnote{\url{https://bitbucket.org/mchaput/whoosh/wiki/Home}} and some sample
documents -- twelve masters theses from the \emph{Ateneo}.  We will need to
develop policies for which documents are permissible for distribution through
this site and work on integrating morphological analysis into the search
engine. Currently, new documents must be approved by an administrator before
being added to the index.

\subsection{Collecting Translations}
The second website will be used by Guarani speakers and learners to produce
translations of relevant documents from Spanish to Guarani or vice-versa.
It will be something like a bilingual wiki, although the interface will
encourage users to edit sentences individually. The software will have found
the sentence boundaries in the initial source-language documents and allow
users to contribute translations of each sentence in turn, while showing the
complete document context.
As a result of this, not only will will we be able to collect bitext training
data, but also useful translations of documents will be produced.

Initially, this site will be seeded with documents from the Spanish and Guarani
Wikipedias. Successful translations of the Spanish-language articles could be
fed back into the Guarani Wikipedia.  Other documents will be added by
Guarani-language educators and perhaps also pulled from Tahekami. Translations
may be assigned as homework by Guarani-language teachers.

The website will keep track of translations contributed by individual users;
there may be game-like features and community voting, where large number of
translations, or particularly good ones, are recognized, perhaps with virtual
prizes and badges. Ideally, I will not spend much time on these particular
features; if we can attract open-source contributors, hopefully they will have
an interest in building it.

We may eventually collect enough bitext with this website such that it makes
sense to develop approaches for determining which sentences are the most
reliable and the most useful for training; this may correlate with quality
judgements from the human volunteers.
Investigating this relationship would make a good research question.

In the medium-term, this website will get an integrated ability to search
a translation memory and automatic suggestions from a machine translation
system\footnote{Features described in a presentation in Spanish here:
\url{http://www.cs.indiana.edu/~gasser/Taller2013/} ; English-language similar
presentation: \url{http://tinyurl.com/alexr-clingding-guarani} }
. While these features will be both useful and present a number of
interesting research questions, they are outside the scope of this
dissertation.
