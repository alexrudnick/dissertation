\section{Acquiring larger bitext corpora}
To gather a larger training corpus, both of Spanish-Guarani bitext and
monolingual Guarani text, we plan to build two websites. One website will be a
collaborative online space for building translations of documents, primarily
from Spanish to Guarani, although the other direction will be useful as well.
The second website will be a repository of Guarani and bilingual documents,
complete with a search engine and the ability to upload new documents.
Initial designs for both of these sites were done as a master's project in HCI
by Alberto Samaniego\footnote{\url{http://albsama.com}}, who will hopefully
continue collaborating on this project from his native Paraguay.

\subsection{Collecting Guarani Documents}
The first website we will develop is called ``Tahekami", which means
\emph{let's search together} in Guarani. It will be a repository of Guarani and
bilingual documents, complete with the ability to upload new documents,
browseable tags, and a search feature that takes into account the morphological
richness of Guarani.

An initial version of this site is already well underway \footnote{It is being
developed at \url{http://github.com/hltdi/gn-documents}}. We have a working
search engine based on the Whoosh library
\footnote{\url{https://bitbucket.org/mchaput/whoosh/wiki/Home}} and some sample
documents -- twelve masters theses from the \emph{Ateneo}.  We will need to
develop policies for which documents are permissible for distribution through
this site and work on integrating morphological analysis into the search
engine. Currently, new documents must be approved by an administrator before
being added to the index.


\subsection{Collecting Translations}
The second website will be used by Guarani speakers and learners to
collaboratively produce translations of relevant documents. Initially, it will
be seeded with documents from the Spanish and Guarani Wikipedias, although
other documents will be added by Guarani-language educators and perhaps also
pulled from Tahekami. Translations may be assigned as homework.

We may eventually collect enough bitext with this website such that it makes
sense to develop approaches for determining which sentences are the most
reliable and the most useful for training; this may correlate with quality
judgements from the human volunteers.
Investigating this relationship would make a good research question.

In the medium-term, this website will get an integrated ability to search
translation memory and automatic suggestions from a machine translation system.
However, these features are outside the scope of this dissertation.

\footnote{Features described in a presentation in Spanish here:
\url{http://www.cs.indiana.edu/~gasser/Taller2013/} . English-language similar
presentation: \url{http://tinyurl.com/alexr-clingding-guarani} }
