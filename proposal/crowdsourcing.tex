\section{Acquiring larger bitext corpora}
To gather a larger Spanish-Guarani bitext corpus, we plan to build a website
where Guarani speakers and learners can collaboratively produce translations
and an online repository of Guarani and bilingual documents.  Here we will need
to develop approaches for determining which examples from the crowdsourced data
are the most reliable and the most useful for training; this may correlate
with quality judgements from the human volunteers, although it is an open
question. Initial designs for the site were done as a master's project in HCI
by Alberto Samaniego, who will soon return to his native Paraguay.

\subsection{Collecting Guarani Documents}
``Tahekami", which in Guarani means \emph{let's search together}.

An initial version of this site is already well underway; we have built a
repository of Guarani documents, complete with the ability to upload new
documents, a search engine, and ...

-- they must be approved by an administrator before being added to
the 

\subsection{Collecting Translations}

In the long run, the latter 
