\section{Acquiring larger bitext corpora}
Guarani is unique among indigenous American languages in that a substantial
number of non-indigenous people speak it. It is spoken by the majority of
Paraguayans, who are likely to be bilingual with Spanish. In practice, many
Paraguayans use a combination of Guarani and Spanish called \emph{Jopar{\'a}}.

To gather a larger Spanish-Guarani bitext corpus, we plan to build a website
where Guarani speakers and learners can collaboratively produce translations
and an online repository of Guarani and bilingual documents.  Here we will need
to develop approaches for determining which examples from the crowdsourced data
are the most reliable and the most useful for training; this may correlate
with quality judgements from the human volunteers, although it is an open
question. Initial designs for the site were done as a master's project in HCI
by Alberto Samaniego, who will soon return to his native Paraguay.

We have made contact with a number of collaborators in Paraguay, including
language activists and educators from the \emph{Ateneo de la Lengua y Cultura
Guaraní} and the \emph{Fundación Yvy Marãe'{\~y}}, schools that offer training
for Guarani-language translators. We have also started discussing development
plans with several local software developers interested in building open source
software.

