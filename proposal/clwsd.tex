\section{Techniques for CL-WSD}

\subsection{Using multilingual evidence}
One of the techniques for CL-WSD that we will investigate is the use of
multiple bitext corpora, so that the software can make use of information from
available bitext from several different language pairs. This approach is
informed by the work of Lefever and Hoste
\cite{lefever-hoste-decock:2011:ACL-HLT2011}, although their technique requires
an entire machine translation system to perform CL-WSD, whereas we consider
CL-WSD to be a subproblem of MT. Thus we would like to perform CL-WSD without
depending on too much additional software infrastructure.

We will develop this technique in a number of variations, including the use of
classifier stacking and graphical models that frame CL-WSD as the problem
of jointly selecting translations into several languages. Earlier this year,
we developed initial versions of this kind of CL-WSD system
\cite{rudnick-liu-gasser:2013:SemEval-2013} and produced some of the best
results in a SemEval shared task on CL-WSD \cite{task10},
translating polysemous nouns from English into other European languages.

\subsection{Lexical selection as a sequence labeling problem}
We will also investigate the use of sequence-labeling models for
lexical selection.  The intuition behind the sequence-labeling approach is that
machine translation implies an ``all-words" WSD task, in that we need to choose
a translation for every word or phrase in the source sentence, and that the
sequence of translations chosen should make sense when taken together.

One promising formalism for this line of work is the Maximum
Entropy Markov Model (MEMM), which can be combined in a straightforward way
with the simpler Hidden Markov Model (HMM). This combination allows for
efficient inference and the ability to trade more computational resources for
richer modeling. More sophisticated sequence models, such as Conditional Random
Fields, may be useful in this task as well.

We will describe our initial experiments in applying these methods to CL-WSD in
both English-Spanish and Spanish-Guarani translation tasks at the HyTra
workshop in August \cite{rudnick-gasser:2013:HyTra-2013}.

\subsection{Applying similar techniques for morphology prediction}
