\urldef{\leydelenguas}\url{http://www.cultura.gov.py/lang/es-es/2011/05/ley-de-lenguas-n%C2%BA-4251/}

\section{Paraguay and the Guarani Language}
Guarani is unique among indigenous American languages in that a substantial
number of non-indigenous people speak it.  The majority of Paraguayans are
conversant in Guarani, although they are likely to be bilingual with Spanish.
In practice, many Paraguayans use a combination of Guarani and Spanish called
\emph{Jopar{\'a}}.

In some official sense, 
Paraguay is a bilingual country -- the 
\emph{Ley de Lenguas} \footnote{\leydelenguas}

Guarani has a rich morphology ...

\begin{itemize}
\item no grammatical gender
\item aspect marked even on nouns
\item polysynthetic
\end{itemize}

We have made contact with a number of collaborators in Paraguay, including
language activists and educators from the \emph{Ateneo de la Lengua y Cultura
Guaraní}
\footnote{\url{http://www.ateneoguarani.edu.py/}}
and the
\emph{Fundación Yvy Marãe'{\~y}} \footnote{\url{http://yvymaraey.org/}},
schools that offer training
for Guarani-language translators.
The language has an enthusiastic community of language educators and activists,
and it features significantly in the sense of national identity and history.

We have also started discussing development plans with several local software
developers -- including some from the local One Laptop Per Child organization
-- interested in building open source software, particularly,
the websites where language learners and activists can help
build larger bilingual corpora, as we will describe next, in Section
\ref{sec:crowdsourcing}.
