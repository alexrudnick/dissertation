\section{CL-WSD for Hybrid Machine Translation in Low-Resource Settings}
For most of the world's language pairs, there are no large bitext corpora
available.
However, once an RBMT or hybrid system is developed for a language pair, it
should be able to make use of any bitext on hand, so that, like SMT, it will
produce better translations as larger corpora become available without
additional code changes.

We will apply the CL-WSD techniques we develop to a number of different MT
systems with different designs, translating several different language pairs.
These will include, at least: a hybrid SCFG-based system that makes use of both
bilingual transfer rules and a monolingual language model, translating from
Spanish to Guarani (Tereré) and a classic transfer-based system translating
Spanish to Quechua (Squoia).
We would also like to integrate into a more sophisticated RBMT system based on
constraint solving and synchronous dependency grammars (L3),
and a second system based on shallow transfer (Apertium), which has been
applied to a large number of language pairs.


(why did we want to cite Hutchins? ...)
\cite{hutchins1992introduction}


\subsection{Tereré}
Since we hope that our ideas about lexical selection will make sense in several
different contexts, we will develop a new machine translation system out of
open-source SMT components, especially relying on the cdec
\footnote{\url{http://cdec-decoder.org}} decoder \cite{Dyer_etal_2010}.

This new system is called Tereré
\footnote{\url{http://github.com/alexrudnick/terere}}.
Tereré will make use of modern hybrid MT techniques; our current design is
fairly similar to the Stat-XFER approach \cite{DBLP:conf/cicling/Lavie08}
developed by researchers at CMU.
Like Stat-XFER, Tereré will make use of bilingual transfer rules, a lexical
transfer stage, and statistical decoding.

%% XXX: be able to explain the difference between SAMT and Stat-XFER.
While the initial transfer rules will likely be written by hand based the
respective grammars of Spanish and Guarani, we may also include
automatically-extracted rules, perhaps via Thrax \cite{weese-EtAl:2011:WMT} or
a forthcoming tool for extracting Inversion Transduction Grammars, from a team
at HKUST \cite{saers-addanki-wu:2013:HyTra}. The use of automatically-extracted
transfer rules would make the system more similar to the SAMT approaches of
Zollmann and Venugopal \shortcite{zollmann-venugopal:2006:WMT}.

We will approach the rich morphology of Guarani and the associated data
sparsity by having the system produce uninflected Guarani stems,
which we will then inflect in a second pass.
In the second pass, we will predict the appropriate morphological features will
with a discriminative sequence-labeling approach based on work at Microsoft
Research \cite{toutanova-suzuki-ruopp:2008:ACLMain}.
Thus both the transfer rules and the language model will be in terms of stemmed
Guarani.

TODO Talk about the other prediction approaches for morphology, like factored
models. Are there others?

Markus also suggests that there will be some cases where we actually just
*know* the right morphology, based on the syntax. Maybe in those cases we can
output an appropriate token to flag to the morphology pass that we'll need to
inflect the output word in a certain way.

%% OK, so then we need to talk about how we're going to use CL-WSD too.
%% XXX: needs expansion
To integrate our CL-WSD system into Tereré, we will automatically produce a
cdec grammar just before decoding, in which a feature is added to each lexical
transfer rule that encodes the preferences of the WSD system.
Then the weights for all of the features in the system can be tuned with MERT
\cite{och:2003:ACL}.

\subsection{SQUOIA}
TODO Add some stuff about Squoia and how we're going to integrate with it.
\cite{riosgonzales-gohring:2013:HyTra}
\footnote{\url{https://code.google.com/p/squoia/}}


SQUOIA is based on the architecture from Matxin \cite{matxin_2005}. 

How are we going to integrate?

\subsection{L3}

\cite{gasser:sxdg}

L3 is an RBMT system based on synchronous dependency grammars and constraint
solving \cite{gasser:sxdg,gasser:aflat2012}.
It depends on linguistic knowledge about the source and target languages and
can also include abstract semantic representations as an intermediate stage in
processing. It integrates morphological analysis and generation for use in
translating morphologically rich languages, such as Guarani.
(The morphological analyzers and generators to be used in Tereré were
originally developed for L3 by Michael Gasser.)

However, L3 is currently entirely rule-based and it needs a better way to rank
the licensed translations of an input sentence. It faces syntactic and lexical
ambiguity both in its analysis of the input sentence and in the construction of
output sentences. Ideally, a good lexical selection module would constrain its
other choices, yielding the higher-quality translations first.

\subsection{Apertium}
Apertium is another RBMT system based on shallow transfer.

\cite{Forcada_theapertium}.
