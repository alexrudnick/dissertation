\section{CL-WSD for Hybrid Machine Translation in Low-Resource Settings}

In recent years, we have seen renewed interest in machine translation systems
that take into account syntactic structure, linguistic knowledge, and semantic
representations.
Hopefully, these will provide better translation for language pairs with
significant reordering or syntactic divergences, and where one or both of the
languages has rich morphology.
The boundaries between rule-based and statistical MT systems are becoming
increasingly blurred, and hybrid systems are being developed in both
directions, with RBMT systems incorporating components based on machine
learning, as well as SMT systems making use of linguistic knowledge for
morphology and syntax.
Additionally, for most of the world's language pairs, there is simply no large
bitext corpus available, so training a purely statistical machine translation
system is infeasible.
Thus, while SMT approaches have had great success, and drastically changed the
machine translation landscape since the 1990s, RBMT approaches are still
relevant for many language pairs.

We would like for RBMT or hybrid systems, once developed, to be able to make
use of any bitext on hand.  Like SMT systems, they should be able to produce
better translations as larger corpora become available, without additional code
changes.

In this dissertation work, we will apply the CL-WSD techniques we develop to a
number of different MT systems with different designs, translating several
different language pairs.  These will include, at least: a hybrid SCFG-based
system that makes use of both bilingual transfer rules and a monolingual
language model, translating from Spanish to Guarani (Tereré) and a classic
transfer-based system translating Spanish to Quechua (Squoia).
We would also like to integrate into a more sophisticated RBMT system based on
constraint solving and synchronous dependency grammars (L3),
and a second system based on shallow transfer (Apertium), which has been
applied to a large number of language pairs.

\subsection{Tereré}
Since we hope that our ideas about lexical selection will make sense in several
different contexts, we will develop a new machine translation system out of
open-source SMT components, particularly relying on the cdec decoder and its
associated tools \cite{Dyer_etal_2010}.

This new system is called ``Tereré"
\footnote{\url{http://github.com/alexrudnick/terere}; 
Tereré is a cold variety of yerba mate brewed with ice water; it is a
specifically Paraguayan specialty.}.
Tereré will make use of modern hybrid MT techniques; our current design is
fairly similar to the Stat-XFER approach \cite{DBLP:conf/cicling/Lavie08}
developed by researchers at CMU.
Like Stat-XFER, Tereré will make use of bilingual transfer rules, a lexical
transfer stage, a target-language LM, and statistical decoding.

While the initial transfer rules will likely be written by hand, based the
respective grammars of Spanish and Guarani, we may also include
automatically-extracted rules, perhaps via Thrax \cite{weese-EtAl:2011:WMT} or
a forthcoming tool for extracting Inversion Transduction Grammars, from Dekai
Wu's team at HKUST \cite{saers-addanki-wu:2013:HyTra}.
The use of automatically-extracted transfer rules would make the system more
similar to the SAMT approaches of Zollmann and Venugopal
\shortcite{zollmann-venugopal:2006:WMT}.

In order to integrate our CL-WSD systems into Tereré, we will automatically
produce a SCFG rules just before decoding, in which features that encodes the
preferences of the WSD system are added to each lexical transfer rule. Then
the weights for all of the features provided to the system (translation
probabilities, LM scores, CL-WSD scores, and perhaps others) can be tuned with
MERT \cite{och:2003:ACL}, and the decoder will use these to search the space of
licensed translations.

We will approach the rich morphology of Guarani and the associated data
sparsity by having the system produce uninflected Guarani stems, which we will
then inflect in a second pass.
In the second pass, we will predict the appropriate morphological features will
with a discriminative sequence-labeling approach based on work at Microsoft
Research \cite{toutanova-suzuki-ruopp:2008:ACLMain}.
Thus both the transfer rules and the language model will be in terms of stemmed
Guarani.
As an alternative, we could adapt the techniques in
\cite{chahuneau:2013:emnlp} to generate translation rules that contain the
appropriately inflected target forms, just before running the decoder.
Rule-based approaches may also be sensible for generating Guarani morphology,
in some cases, and these will have to be investigated.

\subsection{SQUOIA}

TODO Add some stuff about Squoia and how we're going to integrate with it.
\cite{riosgonzales-gohring:2013:HyTra}
\footnote{\url{http://code.google.com/p/squoia/}}


SQUOIA is based on the architecture from Matxin \cite{matxin_2005}. 

How are we going to integrate?

\subsection{L3}

\cite{gasser:sxdg}

L3 is an RBMT system based on synchronous dependency grammars and constraint
solving \cite{gasser:sxdg,gasser:aflat2012}.
It depends on linguistic knowledge about the source and target languages and
can also include abstract semantic representations as an intermediate stage in
processing. It integrates morphological analysis and generation for use in
translating morphologically rich languages, such as Guarani.
(The morphological analyzers and generators to be used in Tereré were
originally developed for L3 by Michael Gasser.)

However, L3 is currently entirely rule-based and it needs a better way to rank
the licensed translations of an input sentence. It faces syntactic and lexical
ambiguity both in its analysis of the input sentence and in the construction of
output sentences. Ideally, a good lexical selection module would constrain its
other choices, yielding the higher-quality translations first.

\subsection{Apertium}
Apertium is another RBMT system based on shallow transfer.

\cite{Forcada_theapertium}.
