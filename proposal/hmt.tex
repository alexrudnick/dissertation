\section{CL-WSD for Hybrid Machine Translation}

(this all needs lots of work)

\subsection{Terere}
Since we hope that our ideas about lexical selection will make sense in several
different contexts, we will also develop a new machine translation system out
of stock SMT components, particularly relying on the
cdec\footnote{\url{http://cdec-decoder.org}} decoder \cite{Dyer_etal_2010}.

This new system is called
Tereré\footnote{\url{http://github.com/alexrudnick/terere}}. Tereré will make
use of modern SMT techniques ...

XXX: be able to explain the difference between SAMT and Stat-XFER.

Talk about hand-written and automatically extracted rules.

How are we going to extract the rules? Probably with Thrax. Alternatively with
Dekai's new thing, once he releases it. It actually doesn't really matter what
we use to extract them.

Talk about prediction for morphology.
\cite{toutanova-suzuki-ruopp:2008:ACLMain}

TODO Talk about the other prediction approaches for morphology, like factored
models. Are there others?

this needs to get much more clear.

We will integrate our CL-WSD systems with RBMT systems, allowing them to be
trained on any available bitext and produce better translations as larger
corpora become available.

%% XXX: still true?
We will initially work with L3, an RBMT system based on synchronous dependency
grammars and constraint solving \cite{gasser:sxdg,gasser:aflat2012};
for comparison we will investigate other RBMT systems as well.
L3 depends on linguistic knowledge about the source and target languages and
can include abstract semantic representations as an intermediate stage in
processing. It also integrates morphological analysis and generation for use in
translating morphologically rich languages, such as Guarani.

However, L3 is currently entirely rule-based and it needs a better way to rank
the possible translations of an input sentence. It faces syntactic and lexical
ambiguity both in its analysis of the input sentence and in the construction of
an output sentence.  A good lexical selection module would constrain its other
choices, yielding higher-quality translations first.

\subsection{CL-WSD for RBMT}
We plan to be able to integrate CL-WSD techniques into two different RBMT
systems, including L3 \cite{gasser:sxdg} and Squoia.

Also possible is Apertium...

TODO Add some stuff about Squoia and how we're going to integrate with it.
\cite{riosgonzales-gohring:2013:HyTra}

\footnote{\url{https://code.google.com/p/squoia/}}

