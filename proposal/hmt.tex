\section{CL-WSD for Hybrid Machine Translation}
In order to develop our ideas about CL-WSD for lexical selection, we will apply
them to a number of different hybrid and RBMT systems with different designs,
including, at least: a hybrid SCFG-based system that makes use of both
bilingual transfer rules and a monolingual language model, and a classic
transfer-based system based on Matxin \cite{matxin_2005}. Time permitting, we
would also like to integrate into L3 \cite{gasser:sxdg}, Apertium
\cite{Forcada_theapertium}.

(why did we want to cite Hutchins? ...)
\cite{hutchins1992introduction}

\subsection{Tereré}
Since we hope that our ideas about lexical selection will make sense in several
different contexts, we will develop a new machine translation system out of
stock SMT components, particularly relying on the cdec
\footnote{\url{http://cdec-decoder.org}} decoder \cite{Dyer_etal_2010}.

This new system is called
Tereré \footnote{\url{http://github.com/alexrudnick/terere}}. Tereré will make
use of modern SMT techniques ...

XXX: be able to explain the difference between SAMT and Stat-XFER.

Talk about hand-written and automatically extracted rules.

How are we going to extract the rules? Probably with Thrax. Alternatively with
Dekai's new thing, once he releases it. It actually doesn't really matter what
we use to extract them.

Talk about prediction for morphology.
\cite{toutanova-suzuki-ruopp:2008:ACLMain}

TODO Talk about the other prediction approaches for morphology, like factored
models. Are there others?

this needs to get much more clear.

We will integrate our CL-WSD systems with RBMT systems, allowing them to be
trained on any available bitext and produce better translations as larger
corpora become available.

\subsection{SQUOIA}
TODO Add some stuff about Squoia and how we're going to integrate with it.
\cite{riosgonzales-gohring:2013:HyTra}
\footnote{\url{https://code.google.com/p/squoia/}}

\subsection{L3}
L3 is an RBMT system based on synchronous dependency grammars and constraint
solving \cite{gasser:sxdg,gasser:aflat2012}.
It depends on linguistic knowledge about the source and target languages and
can also include abstract semantic representations as an intermediate stage in
processing. It integrates morphological analysis and generation for use in
translating morphologically rich languages, such as Guarani.
(The morphological analyzers and generators to be used in Tereré were
originally developed for L3 by Michael Gasser.)

However, L3 is currently entirely rule-based and it needs a better way to rank
the licensed translations of an input sentence. It faces syntactic and lexical
ambiguity both in its analysis of the input sentence and in the construction of
output sentences. Ideally, a good lexical selection module would constrain its
other choices, yielding the higher-quality translations first.

\subsection{Apertium}
Apertium is another RBMT system based on shallow transfer.

\cite{Forcada_theapertium}
