\section{CL-WSD for Hybrid Machine Translation}
In order to develop our ideas about CL-WSD for lexical selection, we will apply
them to a number of different hybrid and RBMT systems with different designs,
including, at least: a hybrid SCFG-based system that makes use of both
bilingual transfer rules and a monolingual language model, and a classic
transfer-based system based on Matxin \cite{matxin_2005}. Time permitting, we
would also like to integrate into L3 \cite{gasser:sxdg}, Apertium
\cite{Forcada_theapertium}.

(why did we want to cite Hutchins? ...)
\cite{hutchins1992introduction}

%% XXX needs some rewording
While there is not much available bitext for training MT systems for most of
the world's language pairs, once an RBMT or hybrid system is brought up, we
would like to allow these systems to make use of any bitext that does become
available, allowing them to, like SMT, produce better translations as larger
corpora become available.

\subsection{Tereré}
Since we hope that our ideas about lexical selection will make sense in several
different contexts, we will develop a new machine translation system out of
stock SMT components, particularly relying on the cdec
\footnote{\url{http://cdec-decoder.org}} decoder \cite{Dyer_etal_2010}.

This new system is called
Tereré \footnote{\url{http://github.com/alexrudnick/terere}}. Tereré will make
use of modern hybrid MT techniques. Our current plan is for it to be fairly
similar to the Stat-XFER approach \cite{DBLP:conf/cicling/Lavie08} developed by
researchers at CMU.
Like Stat-XFER, Tereré will make use of bilingual transfer rules, a lexical
transfer stage, and statistical decoding.

%% XXX: be able to explain the difference between SAMT and Stat-XFER.
While the initial transfer rules will likely be written by hand based the
respective grammars of Spanish and Guarani, we may also include
automatically-extracted rules, perhaps via Thrax \cite{weese-EtAl:2011:WMT} or
a forthcoming tool for extracting Inversion Transduction Grammars, from a team
at HKUST \cite{saers-addanki-wu:2013:HyTra}.

The initial plan for Tereré is to have it produce uninflected Guarani stems,
which we will then take another pass through to inflect.
Thus both the transfer rules and the language model will need to produced in
terms of stemmed Guarani.
Morphological features will then be produced with a sequence-labeling approach
based on discriminative sequence labeling similar to the approached developed
at MSR \cite{toutanova-suzuki-ruopp:2008:ACLMain}.

TODO Talk about the other prediction approaches for morphology, like factored
models. Are there others?

this needs to get much more clear.
OK, so then we need to talk about how we're going to use CL-WSD too.

\subsection{SQUOIA}
TODO Add some stuff about Squoia and how we're going to integrate with it.
\cite{riosgonzales-gohring:2013:HyTra}
\footnote{\url{https://code.google.com/p/squoia/}}

\subsection{L3}
L3 is an RBMT system based on synchronous dependency grammars and constraint
solving \cite{gasser:sxdg,gasser:aflat2012}.
It depends on linguistic knowledge about the source and target languages and
can also include abstract semantic representations as an intermediate stage in
processing. It integrates morphological analysis and generation for use in
translating morphologically rich languages, such as Guarani.
(The morphological analyzers and generators to be used in Tereré were
originally developed for L3 by Michael Gasser.)

However, L3 is currently entirely rule-based and it needs a better way to rank
the licensed translations of an input sentence. It faces syntactic and lexical
ambiguity both in its analysis of the input sentence and in the construction of
output sentences. Ideally, a good lexical selection module would constrain its
other choices, yielding the higher-quality translations first.

\subsection{Apertium}
Apertium is another RBMT system based on shallow transfer.

\cite{Forcada_theapertium}
