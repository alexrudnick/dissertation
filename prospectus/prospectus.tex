\documentclass{article}
%% \usepackage{fullpage}
\usepackage{latexsym}
\usepackage{naaclhlt2013}
\usepackage{url}
\usepackage[utf8]{inputenc}
%% \pagenumbering{gobble}

\title{Cross-lingual Word Sense Disambiguation for Low-Resource Hybrid Machine
Translation}
\author{Alex Rudnick \\
	    Indiana University, School of Informatics and Computing \\
	    {\tt alexr@indiana.edu}}
\date{}

\begin{document}
\maketitle

%% Second, the prospectus is about the PhD dissertation as a body of work, so I
%% wouldn't distinguish between what you've done and future work, instead just
%% describe it in the present (for example, "this dissertation seeks to 
%% demonstrate that...") or the future ("this research will integrate...").

%% Third, "contribution" is normally thought of as research contribution, so I
%% wouldn't include the website or the software in "contributions". I'd add
%% something like "in addition, the research is expected to have several
%% practical outcomes:..."

\section{Overview}

%% notes from Mike
%% First, a dissertation is usually thought of as answering a series of
%% research questions, and these would normally appear at the beginning of the
%% prospectus (and at the beginning of the proposal). So try to frame the
%% problem that way.

%% just to think about it: what are the questions we're answering?
%% - How can we bring up a hybrid MT system for relatively low resource MT?
%% - How can we use CL-WSD techniques to build effective hybrid MT?
%% - How can we collect bigger corpora?
In this dissertation, I will investigate how one can build a hybrid machine
translation (HMT) system for a language pair with relatively modest resources.
I propose that cross-lingual word sense disambiguation (CL-WSD) is a
feasible and practical means for lexical selection in this setting, and work to
demonstrate this in practice by using it to construct a HMT system for
translating between Spanish and Guarani, the co-official languages of Paraguay.
Along the way, I will develop new CL-WSD approaches, including the use of
multilingual evidence and sequence-labeling techniques, and also crowdsourcing
approaches for collecting larger corpora for disadvantaged languages.

Lexical ambiguity presents a serious challenge for rule-based machine
translation (RBMT) systems, since many words have multiple possible
translations in the target language. Moreover, several translations of a given
word may all be syntactically valid in context, but have significantly
different meanings. Even when choosing among near-synonyms, we would like to
respect the selectional preferences of the target language so as to produce
natural-sounding output text.

Writing lexical selection rules by hand is tedious and error-prone; bilingual
informants, if available, may not be able to enumerate the contexts in which
they would choose one alternative over another. Thus we would like to learn
from corpora when possible. However, for most language pairs, suitably large
sentence-aligned bitext corpora are not available, so creating and deploying a
translation system based on machine learning techniques will require collecting
a larger corpus.

%% In this work I will focus on Spanish and Guarani, the co-official languages
%% of Paraguay.
%% I will describe novel approaches to CL-WSD, including using evidence from
%% multilingual sources and sequence labeling techniques. I will also
%% investigate the collection of a larger bilingual corpus through
%% crowdsourcing and the integration of the disambiguation techniques into a
%% rule-based translation engine, producing a hybrid rule-based/statistical
%% translation system.

The major contributions of this work will be new approaches for CL-WSD
and its integration into an RBMT system and, to our knowledge, the first
deployed MT system for the Spanish-Guarani language pair. Additionally, on a
practical level, we will develop a suite of reusable open-source
software, a website where language learners and activists can help build large
bilingual corpora, and a freely available corpus of Spanish-Guarani bitext.

\section{Using multilingual evidence}
One of the techniques for CL-WSD that we will investigate is the use of
multiple bitext corpora, so that the software can make use of information from
available bitext from several different language pairs. This approach is
informed by the work of Lefever and Hoste
\cite{lefever-hoste-decock:2011:ACL-HLT2011}, although their technique requires
an entire machine translation system to perform CL-WSD, whereas we consider
CL-WSD to be a subproblem of MT. Thus we would like to perform CL-WSD without
depending on too much additional software infrastructure.

We will develop this technique in a number of variations, including the use of
classifier stacking and graphical models that frame CL-WSD as the problem
of jointly selecting translations into several languages. Earlier this year,
we developed initial versions of this kind of CL-WSD system
\cite{rudnick-liu-gasser:2013:SemEval-2013} and produced some of the best
results in a SemEval shared task on CL-WSD \cite{task10},
translating polysemous nouns from English into other European languages.

\section{Lexical selection as a sequence labeling problem}
We will also investigate the use of sequence-labeling models for
lexical selection.  The intuition behind the sequence-labeling approach is that
machine translation implies an ``all-words" WSD task, in that we need to choose
a translation for every word or phrase in the source sentence, and that the
sequence of translations chosen should make sense when taken together.

One promising formalism for this line of work is the Maximum
Entropy Markov Model (MEMM), which can be combined in a straightforward way
with the simpler Hidden Markov Model (HMM). This combination allows for
efficient inference and the ability to trade more computational resources for
richer modeling. More sophisticated sequence models, such as Conditional Random
Fields, may be useful in this task as well.

We will describe our initial experiments in applying these methods to CL-WSD in
both English-Spanish and Spanish-Guarani translation tasks at the HyTra
workshop in August \cite{rudnick-gasser:2013:HyTra-2013}.

\section{CL-WSD for Hybrid Machine Translation}
We will integrate our CL-WSD systems with RBMT systems, allowing them
to be trained on any available bitext and produce better translations as larger
corpora become available.
We will initially work with L3, an RBMT system based on synchronous dependency
grammars and constraint solving \cite{gasser:sxdg,gasser:aflat2012},
although for comparison we will investigate other RBMT systems as well. L3
depends on linguistic knowledge about the source and target languages and can
include abstract semantic representations as an intermediate stage in
processing. It also has integrated morphological analysis and generation for
use in translating morphologically rich languages, such as Guarani.

However, L3 is currently entirely rule-based and it needs a better way to rank
the possible translations of an input sentence. It faces syntactic and lexical
ambiguity both in its analysis of the input sentence and in the construction of
an output sentence. As such, it needs an appropriate way to hone in on the most
promising translations first; a good lexical selection module would help
constrain its other choices, guiding it towards high-quality translations
faster.

\section{Acquiring and using larger bitext corpora}
Guarani is unique among indigenous American languages in that a substantial
number of non-indigenous people speak it. It is spoken by the majority of
Paraguayans, who are likely to be bilingual with Spanish. In practice, many
Paraguayans use a combination of Guarani and Spanish called \emph{Jopar{\'a}}.

We plan to build a website where language activists, educators and learners can
collaboratively produce translations between Spanish and Guarani, and a
repository of Guarani and bilingual documents, so that good educational
materials are available for readers. This will also generate an
increasingly large bitext corpus for use in training translation systems. The
initial designs for the site were done as a master's project in HCI by Alberto
Samaniego, who will soon return to his native Paraguay.

We have made contact with a number of collaborators in Paraguay, including
language activists and educators from the \emph{Ateneo de la Lengua y Cultura
Guaraní} and the \emph{Fundación Yvy Marãe'{\~y}}, schools that offer training
for Guarani-language translators. We have also started discussing development
plans with several local software developers interested in building open source
software.

\section{Conclusions}
In this dissertation, I will investigate practical, relatively inexpensive ways
to build hybrid machine translation systems through the application of CL-WSD
techniques and by engaging with a disadvantaged language's activist community
to effectively crowdsource the construction of larger bitext corpora. As a
result, we will deploy an MT system for the co-official languages of Paraguay
and help Guarani speakers develop a repository of educational materials for
their language.

%% Framing the resolution of lexical ambiguities in machine translation as an
%% explicit classification task has a long history; practical work in integrating
%% WSD with statistical machine translation dates back to early SMT work at IBM
%% \cite{Brown91word-sensedisambiguation}, but the problem itself was described in
%% Warren Weaver's prescient 1949 memorandum \cite{warrenmemo}, the fifth section
%% of which outlines the modern conception of word sense disambiguation.
%% More recently, Carpuat and Wu have shown how to use word-sense disambiguation
%% techniques to improve modern phrase-based SMT systems \cite{carpuatpsd}, even
%% though most SMT systems do not use an explicit WSD module \cite{wsdchap3}, as
%% the language model and phrase tables of these systems mitigate lexical
%% ambiguities somewhat.
%% 
%% Treating lexical selection as a word-sense disambiguation problem in
%% which the sense inventory for each source-language word is its set of possible
%% translations is often called cross-lingual WSD (CL-WSD). This framing has
%% received enough attention to warrant shared tasks at recent SemEval workshops;
%% the most recent running of the task is described in \cite{task10}.

\bibliographystyle{naaclhlt2013.bst}
%% \bibliographystyle{plain}
\bibliography{prospectus.bib}{}

\end{document}
