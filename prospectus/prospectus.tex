\documentclass{article}
\usepackage{fullpage}
\usepackage{latexsym}
\usepackage{url}

\title{Cross-lingual Word Sense Disambiguation for Low-Resource Hybrid Machine
Translation}
\author{Alex Rudnick \\
	    Indiana University, School of Informatics and Computing \\
	    {\tt alexr@indiana.edu}}
\date{}

\begin{document}
\maketitle

\begin{abstract}
Here I describe my work, both underway and planned, on cross-lingual word sense
disambiguation and its practical application to a hybrid rule-based/statistical
machine translation system for the Spanish-Guarani language pair. I have
already done some work on using evidence from multilingual sources and
sequence-labeling approaches to address lexical ambiguity in translation. In
future work, I plan to collect more useful bilingual data for our chosen
language pair through crowdsourcing and integrate our disambiguation systems
into our rule-based translation engine, producing a hybrid system.
\end{abstract}

\section{Introduction}
Many words translate ambiguously across languages, and the correct translation
of a given word in the input language typically depends on the context, both
sentence-wide context, document-wide context, and perhaps eventually, on rich
world knowledge.

Lexical selection is the task of selecting the correct words for the 

statistical translation systems often do not use WSD techniques to address the
lexical selection process explicitly, instead making use of language models
trained on the target-language to constrain word choices to ...



Lexical ambiguity presents a serious challenge for rule-based machine
translation (RBMT) systems, since many words have several possible translations
in a given target language, and more than one of them may be syntactically
valid in context.
A translation system must choose a translation for each word or phrase in the
input sentence, and simply taking the most common translation will often fail,
as a word in the source language may have translations in the target language
with significantly different meanings. Even when choosing among near-synonyms,
we would like to respect selectional preferences and common collocations to
produce natural-sounding output text.

Writing lexical selection rules by hand is tedious and error-prone; even if
informants familiar with both languages are available, they may not be able to
enumerate the contexts under which they would choose one translation
alternative over another. Thus we would like to learn from corpora where
possible. 

Framing the resolution of lexical ambiguities in machine translation
as an explicit classification
task has a long history, dating back at least to early SMT work at IBM
\cite{Brown91word-sensedisambiguation}.  More recently, Carpuat and Wu have
shown how to use word-sense disambiguation techniques to improve modern
phrase-based SMT systems \cite{carpuatpsd}, even though the language model and
phrase tables of these systems can mitigate the problem of lexical ambiguities
somewhat. Treating lexical selection as a word-sense disambiguation problem, in
which the sense inventory for each source-language word is its set of possible
translations, is often called cross-lingual WSD (CL-WSD). This framing has
received enough attention to warrant shared tasks at recent SemEval workshops;
the most recent running of the task is described in \cite{task10}.









\section{Using multiple sources of information for cross-language word-sense
disambiguation}
well, we've shown that we can do CL-WSD with graphical models in some sense,
making use of information from multiple parallel corpora...

We have more ideas for improving that approach, though...

\cite{rudnick-liu-gasser:2013:SemEval-2013}


\section{Modeling lexical selection as a sequence tagging problem}

\cite{rudnick-gasser:2013:HyTra-2013}

Intuitively, machine translation implies an ``all-words" WSD task: we need to
choose a translation for every word or phrase in the source sentence, and the
sequence of translations should make sense taken together. Here we begin to
explore CL-WSD not just as a classification task, but as one of sequence
labeling. We describe our approach and implementation, and present two
experiments. In the first experiment, we apply the system to the SemEval 2013
shared task on CL-WSD \cite{task10}, translating from English to Spanish, and
in the second, we perform an all-words labeling task, translating text from the
Bible from Spanish to Guarani. This is work in progress and our code is
currently ``research-quality", but we are developing the software in the
open\footnote{\url{http://github.com/alexrudnick/clwsd}}, with the intention of
using it with free RBMT systems and producing an easily reusable package as the
system matures.


what we want to show with the HMM stuff is that we can do lexical selection
with a very simple model; this will probably not be as good as the richer model
described in the previous section

- but it's really cheap to bring up an all-words system with just HMMs.

- in the future we can expand on the sequence-tagging approach by using CRFs or
  maybe MEMMs (we'll try both, and if there are some other good sequence
  tagging approaches we haven't considered, we'll try those toooo!)



\section{Integrating CL-WSD in Rule-Based Machine Translation Systems}
In L3, we are not only 

\cite{gasser:sxdg}

\cite{gasser:aflat2012}

\section{Acquiring and using larger bitext corpora}

We want to apply these techniques to the concrete problem of translating from
Spanish to Guarani, an indigenous language of South America. Guarani is
disadvantaged in a socio-economic sense, but is broadly spoken in Paraguay. The
majority of Paraguayans are bilingual with Spanish and Guarani, often using a
combination of the two called Jopar{\'a}.

The goal for hybrid rule-based and statistical translation systems, such as L3,
is to make use of the available linguistic knowledge about the source and
target language, in the form of translation rules, but also to make use of any
available training text, either monolingual or bilingual, improving with the
increased corpora size.

Thus we are developing a TAHEKAMI, a platform for helping language teachers and
activists translate text into their preferred language...

\bibliographystyle{plain}
\bibliography{prospectus.bib}{}

\end{document}
