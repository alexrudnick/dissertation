\documentclass{article}
%% \usepackage{fullpage}
\usepackage{latexsym}
\usepackage{naaclhlt2013}
\usepackage{url}
\usepackage[utf8]{inputenc}
%% \pagenumbering{gobble}

\title{Cross-lingual Word Sense Disambiguation for Low-Resource Hybrid Machine
Translation}
\author{Alex Rudnick \\
	    Indiana University, School of Informatics and Computing \\
	    {\tt alexr@indiana.edu}}
\date{}

\begin{document}
\maketitle

\section{Overview}
Here I describe work, both underway and planned, on cross-lingual word sense
disambiguation (CL-WSD) and its practical application to a hybrid
rule-based/statistical machine translation system for the Spanish-Guarani
language pair, the co-official languages of Paraguay.  So far, we have
developed some novel approaches to CL-WSD, using evidence from multilingual
sources and sequence labeling techniques. In future work, I plan to collect
more bilingual data through crowdsourcing, and to integrate the disambiguation
system into our rule-based translation engine, producing a hybrid
rule-based/statistical translation system.

Lexical ambiguity presents a serious challenge for rule-based machine
translation (RBMT) systems, since many words have several possible translations
in the target language. Moreover, many translations of a given word may
be syntactically valid in context, and the alternatives may have significantly
different meanings. Even when choosing among near-synonyms, we would like to
respect the selectional preferences of the target language so as to produce
natural-sounding output text.

Writing lexical selection rules by hand is tedious and error-prone; even if
informants familiar with both languages are available, they may not be able to
enumerate the contexts under which they would choose one alternative over
another. Thus we would like to learn from corpora when possible. However, for
most language pairs, large sentence-aligned bitext corpora are not available,
so creating and deploying a translation system based on machine learning 
techniques will require collecting a larger corpus.

I want to show that cross-lingual word sense disambiguation techniques are a
feasible and practical means for lexical selection in a low-resource machine
translation setting, and to demonstrate this in practice by translating from
Spanish to Guarani. The major contributions of this work will be: (a) new
approaches for CL-WSD and its integration into an RBMT system; (b) a website
where language learners and activists can help build large bilingual corpora;
(c) a suite of reusable open-source software; and (d) to our knowledge, the
first deployed MT system for the Spanish-Guarani language pair.

\section{Using multilingual evidence}
Work is underway on how to use multiple bitext corpora for a CL-WSD task,
allowing us to leverage bitext corpora for language pairs other than the one
for which we are currently translating.

This approach was informed by the work of Lefever and Hoste
\cite{lefever-hoste-decock:2011:ACL-HLT2011}, although their technique requires
an entire machine translation system to perform CL-WSD, whereas we consider
CL-WSD to be a subproblem of MT. Thus we would like to perform CL-WSD without
depending on too much additional software infrastructure.

Earlier this year, we developed three CL-WSD systems
\cite{rudnick-liu-gasser:2013:SemEval-2013}, using classifier stacking and
graphical models to produce some of the best results in a SemEval shared task
on CL-WSD \cite{task10}. In this work we translate from English to other
European languages; it remains to be adapted to translating into Guarani.

\section{Lexical selection as a sequence labeling problem}
We have also started investigating the use of sequence-labeling models for
lexical selection, and will describe our initial experiments at the HyTra
workshop in August \cite{rudnick-gasser:2013:HyTra-2013}. The intuition behind
the sequence-labeling approach is that machine translation implies an
``all-words" WSD task, in that we need to choose a translation for every word
or phrase in the source sentence, and that the sequence of translations chosen
should make sense when taken together.

So far, we have used a combination of Hidden Markov Models and Maximum Entropy
Markov Models for CL-WSD in both English-Spanish and Spanish-Guarani
translation tasks. This will need to be developed further, especially as we
vary the amounts of available training data and the feature sets available to
the classifiers. More sophisticated sequence models, such as Conditional Random
Fields, may be useful in this task as well.

\section{CL-WSD for Hybrid Machine Translation}
We are currently working with L3, a rule-based machine translation system based
on synchronous dependency grammars and constraint solving, developed by Michael
Gasser \shortcite{gasser:sxdg,gasser:aflat2012}. L3 depends on linguistic
knowledge about the source and target languages and can include abstract
semantic representations as an intermediate stage in processing. It also has
integrated morphological analysis and generation for use in translating
morphologically rich languages, such as Guarani.

However, L3 is currently entirely rule-based and it needs a better way to rank
possible translations of an input sentence. It faces syntactic and lexical
ambiguity both in its analysis of the input sentence and in the construction of
an output sentence. As such, it needs an appropriate way to hone in on the most
promising translations first; a good lexical selection module would help
constrain its other choices, guiding it towards high-quality translations
faster.

We plan to integrate our CL-WSD systems into L3 and possibly into other
rule-based MT systems, allowing them to leverage any available bitext corpora
to produce better translations and produce better translations as we collect
more data.

\section{Acquiring and using larger bitext corpora}
We want to apply these techniques to the concrete problem of translating from
Spanish to Guarani, an indigenous language of South America. Guarani is
disadvantaged in a socio-economic sense, but the majority of Paraguayans speak
it, and are likely to be bilingual with Spanish. Guarani is unique among
indigenous American languages in that a substantial number of non-indigenous
people speak it. In practice, many Paraguayans use a combination of Guarani and
Spanish called \emph{Jopar{\'a}}.

We plan to build a website where language activists, educators and learners can
collaboratively produce translations between Spanish and Guarani, and a
repository of Guarani and bilingual documents, so that good educational
materials are available for readers. This will, as a side effect, generate an
increasingly large bitext corpus for use in training translation systems.  The
initial designs of this site were done as an HCI masters project by Alberto
Samaniego, who will soon return to his native Paraguay.

We have made contact with a number of collaborators in Paraguay, including
language activists and educators from the \emph{Ateneo de la Lengua y Cultura
Guaraní} and the \emph{Fundación Yvy Marãe'{\~y}}, schools that offer training
for Guarani-language translators. We have also started discussing development
plans with several local software developers interested in building open source
software.

\section{Conclusions}
For the rest of my doctoral work, I plan to continue development on
cross-lingual word sense disambiguation techniques and their application to
hybrid machine translation, and to develop a practical MT system for
Spanish-Guarani, collecting more data through crowdsourcing as necessary.

%% Framing the resolution of lexical ambiguities in machine translation as an
%% explicit classification task has a long history; practical work in integrating
%% WSD with statistical machine translation dates back to early SMT work at IBM
%% \cite{Brown91word-sensedisambiguation}, but the problem itself was described in
%% Warren Weaver's prescient 1949 memorandum \cite{warrenmemo}, the fifth section
%% of which outlines the modern conception of word sense disambiguation.
%% More recently, Carpuat and Wu have shown how to use word-sense disambiguation
%% techniques to improve modern phrase-based SMT systems \cite{carpuatpsd}, even
%% though most SMT systems do not use an explicit WSD module \cite{wsdchap3}, as
%% the language model and phrase tables of these systems mitigate lexical
%% ambiguities somewhat.
%% 
%% Treating lexical selection as a word-sense disambiguation problem in
%% which the sense inventory for each source-language word is its set of possible
%% translations is often called cross-lingual WSD (CL-WSD). This framing has
%% received enough attention to warrant shared tasks at recent SemEval workshops;
%% the most recent running of the task is described in \cite{task10}.

\bibliographystyle{naaclhlt2013.bst}
%% \bibliographystyle{plain}
\bibliography{prospectus.bib}{}

\end{document}
