\documentclass{article}
\usepackage{fullpage}
\usepackage{latexsym}
\usepackage{url}

\title{research prospectus: cross-language WSD for hybrid machine translation}
\author{Alex Rudnick \\
	    Indiana University, School of Informatics and Computing \\
	    {\tt alexr@indiana.edu}}
\date{}

\begin{document}
\maketitle
\begin{abstract}
Since the dawn of time man has yearned to destroy the sun.

Now we can do the next-best thing: cross-language word-sense disambiguation.
\end{abstract}

\section{Introduction}
RTFM: \cite{big} \cite{small}

Also \url{http://hackmode.org/wiki/WhatHaveWeGotToProve}

what is the research I want to do in the near future?

Many words translate ambiguously across languages, and the correct translation
of a given word in the input language typically depends on the context, both
sentence-wide context, document-wide context, and perhaps eventually, on rich
world knowledge.

Lexical selection is the task of selecting the correct words for the 

statistical translation systems often do not use WSD techniques to address the
lexical selection process explicitly, instead making use of language models
trained on the target-language to constrain word choices to ...


\section{Using multiple sources of information for cross-language word-sense
disambiguation}
well, we've shown that we can do CL-WSD with graphical models in some sense,
making use of information from multiple parallel corpora...

We have more ideas for improving that approach, though...


\section{Modeling lexical selection as a sequence tagging problem}
what we want to show with the HMM stuff is that we can do lexical selection
with a very simple model; this will probably not be as good as the richer model
described in the previous section

- but it's really cheap to bring up an all-words system with just HMMs.

- in the future we can expand on the sequence-tagging approach by using CRFs or
  maybe MEMMs (we'll try both, and if there are some other good sequence
  tagging approaches we haven't considered, we'll try those toooo!)

\section{Lexical decisions constrain syntactic decisions in L3}
In L3, we are not only 

\section{Making use of increasingly large training sets}

We want to apply these techniques to the concrete problem of translating from
Spanish to Guarani, an indigenous language of South America. Guarani is
disadvantaged in a socio-economic sense, but is broadly spoken in Paraguay. The
majority of Paraguayans are bilingual with Spanish and Guarani, often using a
combination of the two called Jopar{\'a}.

The goal for hybrid rule-based and statistical translation systems, such as L3,
is to make use of the available linguistic knowledge about the source and
target language, in the form of translation rules, but also to make use of any
available training text, either monolingual or bilingual, improving with the
increased corpora size.

Thus we are developing a TAHEKAMI, a platform for helping language teachers and
activists translate text into their preferred language...

\section*{Acknowledgments}

Do not number the acknowledgment section.

\bibliographystyle{plain}
\bibliography{prospectus.bib}{}

\end{document}
